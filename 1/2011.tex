\bta{2011}


\begin{enumerate}
	%\renewcommand{\labelenumi}{\arabic{enumi}.}
	% A(\Alph) a(\alph) I(\Roman) i(\roman) 1(\arabic)
	%设定全局标号series=example	%引用全局变量resume=example
	%[topsep=-0.3em,parsep=-0.3em,itemsep=-0.3em,partopsep=-0.3em]
	%可使用leftmargin调整列表环境左边的空白长度 [leftmargin=0em]
	\item
\begin{enumerate}
	%\renewcommand{\labelenumi}{\arabic{enumi}.}
	% A(\Alph) a(\alph) I(\Roman) i(\roman) 1(\arabic)
	%设定全局标号series=example	%引用全局变量resume=example
	%[topsep=-0.3em,parsep=-0.3em,itemsep=-0.3em,partopsep=-0.3em]
	%可使用leftmargin调整列表环境左边的空白长度 [leftmargin=0em]
	\item
氢原子基态的能量为$-13.6V$,那么第一激发态的氢原子电离能为 \xzanswer{B} 

\fourchoices
{13.6 eV}
{3.39 eV}
{7.8 eV}
{10.2 eV}


\item 
普朗克常数$h$的数值为 \xzanswer{D} 

\fourchoices
{$1.05\times10^{-34} $}
{$6.63\times10^{-34}  $}
{$1.05\times10^{-34} \ J \cdot s  $}
{$6.63\times10^{-34} \ J \cdot s $}

\item 
$ A $、$ B $为厄米算符,那么下列各选项为厄米算符的是 \xzanswer{B} 

\fourchoices
{$\dfrac{1}{2}(BA-AB) $}
{$\dfrac{i}{2}(BA-AB) $}
{$AB+iBA $}
{$\dfrac{i}{2}(BA+AB)$}

\item 
对于中心力场,下列各式正确的是 \xzanswer{C} 


\fourchoices
{$\int_{0}^{\infty} \mu^{2}(r) r^{2} d r=1$}
{$ \int_{0}^{\infty} \mu^{2} (r) 4 \pi r^{2} d r=1$}
{$\int_{0}^{\infty} R_{l}^{2}(r) r^{2} d r=1$}
{$\int_{0}^{\infty} R_{l}^{2}(r) 4 \pi r^{2} d r=1$}




其中:$\mu{(r)}=R_l(r)r$

\item 经典力学中有$\vec{L}=\vec{r}\times\vec{p}=-\vec{p}\times \vec{r}$,那么在量子力学中$\vec{L}=\vec{r}\times\vec{p}=-\vec{p}\times \vec{r}$是否也成立。请说明理由。


\banswer{
	成立,写成分量形式可证
}


\item 
在$(\vec{S}^2,S_z)$的共同的本征态$ Y_{10} $中,写出$S_x,S_y$的矩阵表示,并说明是否可以找到这样的一个表象,使得$S_x,S_y,S_z$在该表象中的矩阵表示均为实矩阵,并说明理由。

\banswer{
	%$S_{x}=\frac{\hbar}{2}\left(\begin{array}{ll}0 & 1 \\ 1 & 0\end{array}\right), S_{y}=\frac{\hbar}{2}\left(\begin{array}{cc}0 & -i \\ i & 0\end{array}\right)$,不存在一个表象使得 $S_{x}, S_{y}, S_{z}$ 在该表象中的矩阵表示均为实矩阵。
\iffalse
矩阵元为\\
\[ 
	\begin{aligned}
	<l ^{\prime} m ^{\prime} |S_{x}|lm>=  \frac{1}{2}\left(
	\sqrt{l(l+1)-m(m+1)} \cdot \delta_{l ^{\prime} l} \delta_{m ^{\prime} ,m+1} +\right.\\
 \left.\sqrt{l(l+1)-m(m-1)} \cdot \delta_{l ^{\prime} l} \delta_{m ^{\prime} ,m-1}
	\right) 
\end{aligned}
 \]
\[ 
\begin{aligned}
<l ^{\prime} m ^{\prime} |S_{y}|lm>=\frac{1}{2i}\left(
\sqrt{l(l+1)-m(m+1)} \cdot \delta_{l ^{\prime} l} \delta_{m ^{\prime} ,m+1} +\right.\\
\left. \sqrt{l(l+1)-m(m-1)} \cdot \delta_{l ^{\prime} l} \delta_{m ^{\prime} ,m-1}
\right)
\end{aligned}
\] 
\fi
可用升降算符写出矩阵元,再写出矩阵。或者直接套用泡利矩阵的形式
\[ 
S_{x}=\hbar\left(\begin{array}{ll}
	0 & 1 \\
	1 & 0
\end{array}\right) \quad 
S_{y}=\hbar\left(\begin{array}{cc}
	0 & -i \\
	i & 0
\end{array}\right)  \quad 
S_{z}=\hbar\left(\begin{array}{cc}
	1 & 0 \\
	0 & -1
\end{array}\right)
 \]
 不存在,因为实矩阵不可能使得对易关系$ [S_{x}, S_{y}]=i\hbar S_{z} $成立
}


\item 
写出氢原子、一维简谐振子、一维无限深势阱的能级,并用示意图表示。

\banswer{
	氢原子: $E_{n}=-\frac{e^{2}}{2 a} \frac{1}{n^{2}}\left(a=\frac{\hbar^{2}}{m e^{2}}\right)$\\一维简谐振子: $E_{n}=\left(n+\frac{1}{2}\right) \hbar \omega$\\ 一维无限深势阱: $E_{n}=\frac{n^{2} \pi^{2} \hbar^{2}}{2 m a^{2}}$\\
	图略
}


\item 
两个非全同粒子处于态$\psi(x_1,x_2)$,求出一个粒子处于$p_{1}^{\prime}, p_{1}^{\prime \prime}$之间,另一个粒子处于$x_{2}^{\prime}, x_{2}^{\prime \prime}$之间的几率。

\banswer{
	$P=\frac{\int_{x_{2}^{\prime}}^{x_{2}^{\prime \prime}} \int_{p_{1}^{\prime}}^{p_{1}^{\prime \prime}} \left|\frac{1}{\sqrt{2 \pi \hbar}} \int_{-\infty}^{+\infty} \psi\left(x_{1}, x_{2}\right) e^{-\frac{i x_{1} p}{\hbar}} d x_{1}\right|^{2} d p_{1} d x_{2}}{\int_{-\infty}^{+\infty} \int_{-\infty}^{+\infty}\left|\psi\left(x_{1}, x_{2}\right)\right|^{2} d x_{1} d x_{2}}$
}


\end{enumerate}


\item 	已知$\hat{p}_r=\frac{1}{2}(\frac{\vec{r}}{r}\cdot\vec{p}+\vec{p}\cdot\frac{\vec{r}}{r})$
\begin{enumerate}
	%\renewcommand{\labelenumi}{\arabic{enumi}.}
	% A(\Alph) a(\alph) I(\Roman) i(\roman) 1(\arabic)
	%设定全局标号series=example	%引用全局变量resume=example
	%[topsep=-0.3em,parsep=-0.3em,itemsep=-0.3em,partopsep=-0.3em]
	%可使用leftmargin调整列表环境左边的空白长度 [leftmargin=0em]
	\item
$\hat{p}_r$是否为厄米算符,为什么?
\item 
写出$\hat{p}_r$的算符表示。
\item 
求出$[\hat{r},\hat{p}_r]=$?
\end{enumerate}


\banswer{
\begin{enumerate}
	%\renewcommand{\labelenumi}{\arabic{enumi}.}
	% A(\Alph) a(\alph) I(\Roman) i(\roman) 1(\arabic)
	%设定全局标号series=example	%引用全局变量resume=example
	%[topsep=-0.3em,parsep=-0.3em,itemsep=-0.3em,partopsep=-0.3em]
	%可使用leftmargin调整列表环境左边的空白长度 [leftmargin=0em]
	\item
	是厄米算符,易证
	\item 
	$\hat{p}_{r}=-i \hbar\left(\frac{1}{r}+\frac{\partial}{\partial r}\right)$
	\item 
	$\left[r, p_{r}\right]=i \hbar$
\end{enumerate}

	
}



\newpage
\item 
($30$)有一质量为$m$的粒子在半径为$R$的圆周上运动,现加一微扰:
$$
H'=V(\varphi)=\begin{cases}
V_1,&-\alpha<\varphi<0;\\
V_2,&0<\varphi<\alpha;\\
0,     &\text{其他}
\end{cases}
$$
其中$\alpha<\pi$,求对最低两能级的一级修正。

\banswer{
	哈密顿算符$ \hat{H}_{0}=\frac{l_{z}^{2}}{2I} $,$ \Psi_{n}^{(0)}=\frac{1}{\sqrt{2\pi}}  e^{im\phi} $,$ E_{0}^{(0)}=\frac{m^{2}\hbar^{2}}{2I} $\\
	$ n=0 $时无简并,$ E_{0}=(v_{1}+v_{2}) \alpha  $\\
	$ n=1 $时二重简并,分裂成两条,\\
	$ E_{11}=\frac{\hbar^{2}}{2I}+ (v_{1}+v_{2}) \alpha + 2\sin  \alpha \sqrt{v_{1}^{2} +v_{2}^{2}+2v_{1}v_{2}\cos 2 \alpha  }$\\
	$ E_{12}=\frac{\hbar^{2}}{2I}+ (v_{1}+v_{2}) \alpha - 2\sin  \alpha \sqrt{v_{1}^{2} +v_{2}^{2}+2v_{1}v_{2}\cos 2 \alpha  }$\\
	其中$ I=mR^{2} $
}


\item 
($30$)一粒子在一维无限深方势阱$(0<x<a)$中运动,时间$t=0$时处在基态。此时加入一个高为$V_0$,宽为$b(b\ll a)$,中心在$\frac{a}{2}$的方势垒微扰。求$t_0(t_0>0)$时撤去微扰,体系处于前三个激发态的概率。


\banswer{
\begin{equation*}\label{key}
	\begin{aligned}
		<m|H ^{\prime} |1>&=\frac{2}{a} \cdot V_{0} \cdot \int_{\frac{a-b}{2}}^{\frac{a+b}{2}} \sin \frac{m\pi}{a} x \sin \frac{\pi}{a} x dx \\
		&=\frac{2}{\pi} \cdot V_{0} \cdot \int_{\frac{\pi}{2} -\frac{\pi}{2} \frac{b}{a} }^{\frac{\pi}{2} +\frac{\pi}{2} \frac{b}{a} } \sin m\theta \sin \theta d\theta \\
		& \approx  \frac{2}{\pi} \cdot V_{0} \cdot \frac{1}{m} \cdot \cos m\theta \Big| _{\frac{\pi}{2} -\frac{\pi}{2} \frac{b}{a} }^{\frac{\pi}{2} +\frac{\pi}{2} \frac{b}{a} }
	\end{aligned}
\end{equation*}
这里,很容易得到$ \braket{2|H ^{\prime} |1}=0 $,$ \braket{3|H ^{\prime} |1}=\frac{2}{\pi} V_{0}  \frac{ -2 }{ 3 } \sin (\frac{3}{2}\pi\frac{b}{a} ) \approx -2 \frac{b}{a} V_{0} $,$ \braket{4|H ^{\prime} |1}=0 $。这里也可以采用和差化积的方式来进行精确的计算
\[ 
<m|H ^{\prime} |1>=\frac{V_{0}}{\pi} \left(\frac{1}{m-1} \sin(m-1)t -\frac{1}{m+1} \sin(m+1)t  \right)  \Big| _{\frac{\pi}{2} -\frac{\pi}{2} \frac{b}{a} }^{\frac{\pi}{2} +\frac{\pi}{2} \frac{b}{a} }
 \]
得到的结果除$ \braket{3|H ^{\prime} |1}=-\frac{V_{0}}{\pi} \left(\sin \pi\frac{b}{a} +\frac{1}{2} \sin 2\pi \frac{b}{a} \right)\approx -2\frac{b}{a} V_{0} $外其余一样,不过也可以看到取近似后两者的结果是一致的。
	
$ P_{1\rightarrow 2}=0 $,$ P_{1\rightarrow 3}=\left|  \frac{H ^{\prime} _{31}}{\hbar \omega_{31}}\right| ^ 2 \cdot \left( 1- \cos \omega_{31} t_{0} \right) $,$ P_{1\rightarrow 4}=0 $
}


\item 
在$(l^2,l_z)$的共同表象中,粒子处于$Y_{20}$态,求$l_x$的可能值及相应的几率。

\banswer{ 
$ l_{x} $的可能测值及相应的概率为:
\begin{center}
 \begin{tabular}{|c|c|c|c|c|c|}
 \hline 
$l_{x}$ &$2 \hbar $ & $ \hbar$ & $0$ & $-\hbar$ & $-2 \hbar$ \\
\hline
$ P $ &$\frac{3}{8}$ & $0$ &$\frac{1}{4}$  & $0$ & $ \frac{3}{8}$  \\ 
 \hline 
 \end{tabular}
\end{center}
解法有几种。共五个概率值,五个未知数,找五个独立方程即可,找方程的过程中,可充分利用$ l_{z}\ket{10}=0 $的性质,用$ l_{x} $本征态的升降算符来表示$ l_{z} $。
$ l_{x} $本征值为$ m $的本征态是$ e^{-i\frac{\pi}{2}l_{y}}\ket{ lm} $,也可以推导相应矩阵元$\bra{lm ^{\prime} } e^{-i\frac{\pi}{2}l_{y}}\ket{ lm} $的关系式。
}



\end{enumerate}






