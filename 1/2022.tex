\bta{2022}

\begin{enumerate}
	%\renewcommand{\labelenumi}{\arabic{enumi}.}
	% A(\Alph) a(\alph) I(\Roman) i(\roman) 1(\arabic)
	%设定全局标号series=example	%引用全局变量resume=example
	%[topsep=-0.3em,parsep=-0.3em,itemsep=-0.3em,partopsep=-0.3em]
	%可使用leftmargin调整列表环境左边的空白长度 [leftmargin=0em]
	\item
\begin{enumerate}
	%\renewcommand{\labelenumi}{\arabic{enumi}.}
	% A(\Alph) a(\alph) I(\Roman) i(\roman) 1(\arabic)
	%设定全局标号series=example	%引用全局变量resume=example
	%[topsep=-0.3em,parsep=-0.3em,itemsep=-0.3em,partopsep=-0.3em]
	%可使用leftmargin调整列表环境左边的空白长度 [leftmargin=0em]
	\item
 已知 $[\hat{A}, \hat{B}]=C$,其中$C  $ 为常数, $\hat{B}$ 为厄米算符。 求 $\left[\hat{A}, e^{\hat{B}}\right]$ 的值, 并说明在什么条 件下, $\left[\hat{A}, e^{\hat{B}}\right]$ 是厄米算符 ?
	

	
\item 
若$ \hat{A} $、$ \hat{B} $ 为厄米算符, 判断以下是否为厄米算符, 并说明理由。
\begin{enumerate}
	%\renewcommand{\labelenumi}{\arabic{enumi}.}
	% A(\Alph) a(\alph) I(\Roman) i(\roman) 1(\arabic)
	%设定全局标号series=example	%引用全局变量resume=example
	%[topsep=-0.3em,parsep=-0.3em,itemsep=-0.3em,partopsep=-0.3em]
	%可使用leftmargin调整列表环境左边的空白长度 [leftmargin=0em]
	\item
	$\hat{A}+i \hat{B}$
	\item 
	$(\hat{A}+i \hat{B})(\hat{A}-i \hat{B})$
	\item 
	$(\hat{A}+\hat{B})^{n}$
	\item 
	$(\hat{A}+\hat{B})(\hat{A}-\hat{B})$
	
	
	
\end{enumerate}

	
\end{enumerate}


\banswer{
	
}


\item
有一质量为 $m$ 的粒子处在$V_{1}(x)=\frac{1}{2} k x^{2}$的一维势场中, 并处于$ V_{1}(x) $的基态。
\begin{enumerate}
	%\renewcommand{\labelenumi}{\arabic{enumi}.}
	% A(\Alph) a(\alph) I(\Roman) i(\roman) 1(\arabic)
	%设定全局标号series=example	%引用全局变量resume=example
	%[topsep=-0.3em,parsep=-0.3em,itemsep=-0.3em,partopsep=-0.3em]
	%可使用leftmargin调整列表环境左边的空白长度 [leftmargin=0em]
	\item
	若在 $t$ 时刻$ V_{1}(x) $突然变为 $V_{2}(x)=k x^{2}$, 求$ t $时刻粒子处于 $V_{2}(x)$ 基态的概率。
	\item 
	 若在 $t=0$ 时刻, 粒子所处的势场突然由 $V_{1}(x)$ 变为 $V_{2}(x)$, 经历时间 $\tau$ 后, 势场又突然变回 $V_{1}(x)$, 问$ \tau $取什么值时粒子刚好恢复到势场 $V_{1}$ 的基态。
	
\end{enumerate}

\banswer{
	参见陈鄂生 1.19, 1.20。
}




\item 
有一质量为 $m$ 的粒子在长度为 $L$ 的圆环上自由运动:
\begin{enumerate}
	%\renewcommand{\labelenumi}{\arabic{enumi}.}
	% A(\Alph) a(\alph) I(\Roman) i(\roman) 1(\arabic)
	%设定全局标号series=example	%引用全局变量resume=example
	%[topsep=-0.3em,parsep=-0.3em,itemsep=-0.3em,partopsep=-0.3em]
	%可使用leftmargin调整列表环境左边的空白长度 [leftmargin=0em]
	\item
	求定态波函数和能级。
	\item 
	若引入微扰 $H^{\prime}=-V_{0} \exp \left(-\frac{x^{2}}{a^{2}}\right) $,其中$0<x<L$,$ a<<L$, 求能级的一级修正。
	
	
\end{enumerate}

\banswer{
	
}



\newpage
\item 
一自旋为 $\frac{1}{2}$ 的粒子处于 $x$ 方向的磁场中, 其哈密顿量为 $H=-\gamma B S_{x}$,在$ t=0$ 时刻, 粒子处于 $S_{\mathrm{z}}=+\frac{1}{2}$ 的态。
\begin{enumerate}
	%\renewcommand{\labelenumi}{\arabic{enumi}.}
	% A(\Alph) a(\alph) I(\Roman) i(\roman) 1(\arabic)
	%设定全局标号series=example	%引用全局变量resume=example
	%[topsep=-0.3em,parsep=-0.3em,itemsep=-0.3em,partopsep=-0.3em]
	%可使用leftmargin调整列表环境左边的空白长度 [leftmargin=0em]
	\item
求粒子能量的本征函数和本征值。
\item 
 求在 $t$ 时刻测得 $S_{\mathrm{z}}=+\frac{1}{2}$ 的概率。
\item 
若在时间 $\frac{t}{2}$ 与 $t$ 相继两次测量粒子的自旋, 求两次测得 $S_{\mathrm{z}}$ 分量均为 $+\frac{1}{2}$ 的概率。
	
\item 
若在 $\frac{t}{N}, \frac{2 t}{N}, \frac{3 t}{N}, \cdots, t$ 相继进行$ N $次测量粒子的自旋, 求$ N $次测得 $S_{\mathrm{z}}$ 分量均为 $+\frac{1}{2}$ 的概率。 当 $N \rightarrow +\infty$ 时, 求其极限。	
	
\end{enumerate}


\banswer{
	
}



\item 
有一质量为 $m$ 的粒子在电磁场中运动,哈密顿量为 $H=\frac{1}{2 m}(\vv{p}-q \vv{\mathrm{A}})^{2}+q \varphi$, 其中有
\[ 
\vv{E}=-\vv{\nabla} \varphi-\frac{\partial \vv{\mathrm{A}}}{\partial t}  \qquad \vv{B}=\vv{\nabla} \times \vv{\mathrm{A}}
 \]
\begin{enumerate}
	%\renewcommand{\labelenumi}{\arabic{enumi}.}
	% A(\Alph) a(\alph) I(\Roman) i(\roman) 1(\arabic)
	%设定全局标号series=example	%引用全局变量resume=example
	%[topsep=-0.3em,parsep=-0.3em,itemsep=-0.3em,partopsep=-0.3em]
	%可使用leftmargin调整列表环境左边的空白长度 [leftmargin=0em]
	\item
	求$\frac{d\braket{\vv{r}}}{d t}$, 其中 $\braket{\vv{r}}$ 为半径 $\vv{r}$ 的平均值。
	
	\item 
	若定义 $\braket{\vv{v}}=m \frac{d\braket{\vv{r}}}{d t}$, 写出粒子处于均匀电场 $ \vv{E} $ 和均匀磁场 $ \vv{B}  $中的 $\frac{d\langle\vv{v}\rangle}{d t}$ 表达式。(将结果写成类似洛伦兹力的形式)
	
\end{enumerate}

\banswer{
参见量子力学曾书课本电磁场一章 或者 参见格里菲斯第二版 4.59
}

 
 
\end{enumerate}

