

\bta{2009}

\begin{enumerate}
	%\renewcommand{\labelenumi}{\arabic{enumi}.}
	% A(\Alph) a(\alph) I(\Roman) i(\roman) 1(\arabic)
	%设定全局标号series=example	%引用全局变量resume=example
	%[topsep=-0.3em,parsep=-0.3em,itemsep=-0.3em,partopsep=-0.3em]
	%可使用leftmargin调整列表环境左边的空白长度 [leftmargin=0em]
	\item
($30'$)已知在$(l^2,l_z)$的表象中,
$$l_x=\frac{\hbar}{\sqrt2}\begin{vmatrix}0&1&0\\1&0&1\\0&1&0\end{vmatrix}$$
求:
\begin{enumerate}
	%\renewcommand{\labelenumi}{\arabic{enumi}.}
	% A(\Alph) a(\alph) I(\Roman) i(\roman) 1(\arabic)
	%设定全局标号series=example	%引用全局变量resume=example
	%[topsep=-0.3em,parsep=-0.3em,itemsep=-0.3em,partopsep=-0.3em]
	%可使用leftmargin调整列表环境左边的空白长度 [leftmargin=0em]
	\item
$l_x$的本征值和相应的本征函数;

\item 
$l_y$的矩阵表示。
\end{enumerate}

\banswer{
	
}


\item 
已知一粒子处于一维谐振子势场中运动,势能为$V(x)=\frac{1}{2}kx^2(k>0)$,求:
\begin{enumerate}
	%\renewcommand{\labelenumi}{\arabic{enumi}.}
	% A(\Alph) a(\alph) I(\Roman) i(\roman) 1(\arabic)
	%设定全局标号series=example	%引用全局变量resume=example
	%[topsep=-0.3em,parsep=-0.3em,itemsep=-0.3em,partopsep=-0.3em]
	%可使用leftmargin调整列表环境左边的空白长度 [leftmargin=0em]
	\item
粒子的基态本征函数$\psi_0(x)$;

\item 
若势场突然变为$V'(x)=kx^2$,则粒子仍然处于基态的概率。
\end{enumerate}

提示:用湮灭算符$\hat{a}_-=\sqrt{\frac{m\omega}{2\hbar}}(\hat{x}+\frac{i}{m\omega}\hat{p}),\sqrt2=1.414,\sqrt[4]{2}=1.189$。


\banswer{
	
}



\item 	
($30'$)若已知$[\hat{a}_i,\hat{a}_j]=[\hat{a}_i^\dagger,\hat{a}_j^\dagger]=0,[\hat{a}_i,\hat{a}_j^\dagger]=\delta_{ij}$,其中$i,j=1,2$。设$J_x=\frac{1}{2}(\hat{a}^\dagger_1\hat{a}_2+\hat{a}^\dagger_2\hat{a}_1),J_y=\frac{i}{2}(\hat{a}^\dagger_1\hat{a}_2-\hat{a}^\dagger_2\hat{a}_1),J_z=\frac{1}{2}(\hat{a}^\dagger_2\hat{a}_2-\hat{a}^\dagger_1\hat{a}_1)$,求:
\begin{enumerate}
	%\renewcommand{\labelenumi}{\arabic{enumi}.}
	% A(\Alph) a(\alph) I(\Roman) i(\roman) 1(\arabic)
	%设定全局标号series=example	%引用全局变量resume=example
	%[topsep=-0.3em,parsep=-0.3em,itemsep=-0.3em,partopsep=-0.3em]
	%可使用leftmargin调整列表环境左边的空白长度 [leftmargin=0em]
	\item
$J_x,J_y,J_z$的关系式;

\item 
$J^2=J_x^2+J_y^2+J_z^2$,试用$\hat{a}_1,\hat{a}^\dagger_1,\hat{a}_2,\hat{a}_2^\dagger$表示$J^2$。
\end{enumerate}

\banswer{
	
}


\newpage
\item 
($30'$)已知两种中微子的本征态为$|V_1\rangle$和$|V_2\rangle$ ,能量本征值为$E=pc+\frac{m_i^2c^4}{pc}$(其中$i=1,2$),电子中微子的本征态为$|V_e\rangle=\cos\theta|V_1\rangle+\sin\theta|V_2\rangle$,$\mu$子中微子的本征态为$|V_\mu\rangle=-\sin\theta|V_1\rangle+\cos\theta|V_2\rangle$,其中$\theta$是混合角。某体系中在$t=0$时,电子中微子处于态$|V_e\rangle$,求:
\begin{enumerate}
	%\renewcommand{\labelenumi}{\arabic{enumi}.}
	% A(\Alph) a(\alph) I(\Roman) i(\roman) 1(\arabic)
	%设定全局标号series=example	%引用全局变量resume=example
	%[topsep=-0.3em,parsep=-0.3em,itemsep=-0.3em,partopsep=-0.3em]
	%可使用leftmargin调整列表环境左边的空白长度 [leftmargin=0em]
	\item
$t$时刻中微子所处的状态;

\item 
$t$时刻电子中微子处于基态的概率。
	
\end{enumerate}


\banswer{
	
}


\item 
($30'$)设在氚核中,质子和中子的作用表示成$V(r)=-V_0e^{-\frac{r}{a}}$,试用$\psi=e^{-\frac{\lambda r}{2a}}$($\lambda$为变数)为试探波函数,以变分法求:
\begin{enumerate}
	%\renewcommand{\labelenumi}{\arabic{enumi}.}
	% A(\Alph) a(\alph) I(\Roman) i(\roman) 1(\arabic)
	%设定全局标号series=example	%引用全局变量resume=example
	%[topsep=-0.3em,parsep=-0.3em,itemsep=-0.3em,partopsep=-0.3em]
	%可使用leftmargin调整列表环境左边的空白长度 [leftmargin=0em]
	\item
基态能量的近似值;

\item 
若$V_0=32.7 \ \mathrm{Mev} ,a=2.16 \ \mathrm{fm}$,试确定$\lambda$的值。

\end{enumerate}

\banswer{
	
}



\end{enumerate}

