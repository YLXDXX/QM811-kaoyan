
\bta{2007A}
\begin{enumerate}
	%\renewcommand{\labelenumi}{\arabic{enumi}.}
	% A(\Alph) a(\alph) I(\Roman) i(\roman) 1(\arabic)
	%设定全局标号series=example	%引用全局变量resume=example
	%[topsep=-0.3em,parsep=-0.3em,itemsep=-0.3em,partopsep=-0.3em]
	%可使用leftmargin调整列表环境左边的空白长度 [leftmargin=0em]
	\item
($30'$)在一维无限深方势阱$(0<x<a)$中运动的粒子受到微扰
$$H'(x)=\begin{cases}0,&0<x<\frac{a}{3},\frac{2a}{3}<x<a\\
-V_1,&\frac{a}{3}<x<\frac{2a}{3}\end{cases}$$
作用。试求基态能量的一级修正。


\banswer{
	
}


\item 
($30'$)粒子在势场$V(x)$中运动并处于束缚定态$\psi_n(x)$中。试证明粒子所受势场的作用力的平均值为$ 0  $。


\banswer{
	
}


\item 
($30'$)
\begin{enumerate}
	%\renewcommand{\labelenumi}{\arabic{enumi}.}
	% A(\Alph) a(\alph) I(\Roman) i(\roman) 1(\arabic)
	%设定全局标号series=example	%引用全局变量resume=example
	%[topsep=-0.3em,parsep=-0.3em,itemsep=-0.3em,partopsep=-0.3em]
	%可使用leftmargin调整列表环境左边的空白长度 [leftmargin=0em]
	\item
考虑自旋为$\frac{1}{2}$的系统。试在$(\hat{S}^2,\hat{S}_z)$表象中求算符$A\hat{S}_y+B\hat{S}_z$的本征值及归一化的本征态。其中$\hat{S}_y,\hat{S}_z$是自旋角动量算符,而$A,B$为实常数。

\item 
假定此系统处于以上算符的一个本征态上,求测量$\hat{S}_y$得到结果为$\frac{\hbar}{2}$的概率。
\end{enumerate}


\banswer{
	
}


\newpage
\item 
($30'$)
两个无相互作用的粒子置于一维无限深势阱中$(0<x<a)$中,对下列两种情况写出两粒子体系具有的两个最低总能量,相应的简并度以及上述能级对应的所有二粒子波函数。
\begin{enumerate}
	%\renewcommand{\labelenumi}{\arabic{enumi}.}
	% A(\Alph) a(\alph) I(\Roman) i(\roman) 1(\arabic)
	%设定全局标号series=example	%引用全局变量resume=example
	%[topsep=-0.3em,parsep=-0.3em,itemsep=-0.3em,partopsep=-0.3em]
	%可使用leftmargin调整列表环境左边的空白长度 [leftmargin=0em]
	\item
两个自旋为$\frac{1}{2}$的可区分粒子;

\item 
两个自旋为$\frac{1}{2}$的全同粒子。

\end{enumerate}


\banswer{
	
}



\item 
($30'$)一个质量为$m$的粒子被限制在$r=a$和$r=b$的两个不可穿透的同心球面之间运动。不存在其它势,求粒子的基态能量和归一化波函数。(1996 年第四题) 

\banswer{
	
}



\end{enumerate}


