\bta{2019}

\begin{enumerate}
	%\renewcommand{\labelenumi}{\arabic{enumi}.}
	% A(\Alph) a(\alph) I(\Roman) i(\roman) 1(\arabic)
	%设定全局标号series=example	%引用全局变量resume=example
	%[topsep=-0.3em,parsep=-0.3em,itemsep=-0.3em,partopsep=-0.3em]
	%可使用leftmargin调整列表环境左边的空白长度 [leftmargin=0em]
	\item
设$ 1 $维线性谐振子的初始状态为\(\psi(x, 0)=A\left[3 \psi_{0}(x)+4 \psi_{1}(x)\right]\),其中
	\(\psi_{0}(x), \psi_{1}(x)\)分别为体系的归一化的基态和第一激发态的波函数。
	\begin{enumerate}
		%\renewcommand{\labelenumi}{\arabic{enumi}.}
		% A(\Alph) a(\alph) I(\Roman) i(\roman) 1(\arabic)
		%设定全局标号series=example	%引用全局变量resume=example
		%[topsep=-0.3em,parsep=-0.3em,itemsep=-0.3em,partopsep=-0.3em]
		%可使用leftmargin调整列表环境左边的空白长度 [leftmargin=0em]
		\item
求归一化常数$ A $。
	
	\item 
	计算\(\psi(x, t)\)、\(|\psi(x, t)|^{2}\)。
	
	\item 
	求粒子能量的可取值,各可取值出现的概率,及能量的平均值。
		\end{enumerate}

\banswer{
	
}


\item 
设体系由$ 3 $个无相互作用粒子组成,$ 3 $个粒子分别处于正交归一单粒子态\(\psi_a\),\(\psi_b\),\(\psi_c\)。
\begin{enumerate}
	%\renewcommand{\labelenumi}{\arabic{enumi}.}
	% A(\Alph) a(\alph) I(\Roman) i(\roman) 1(\arabic)
	%设定全局标号series=example	%引用全局变量resume=example
	%[topsep=-0.3em,parsep=-0.3em,itemsep=-0.3em,partopsep=-0.3em]
	%可使用leftmargin调整列表环境左边的空白长度 [leftmargin=0em]
	\item
若$ 3 $个粒子为自旋为$ 0 $的非全同玻色子,求出其波函数。
	
	\item 
	若$ 3 $个粒子为自旋为$ 0 $的全同玻色子,求出其波函数。
	
	\item 
	若$ 3 $个粒子为自旋$ 1/2 $的全同费米子,自旋处于\(|\uparrow\rangle\),求出其波函数。
\end{enumerate}


\banswer{
	
}


\item 
设粒子的轨道角动量量子数\(l=1\)(取\(\hbar=1\))。
	\begin{enumerate}
		%\renewcommand{\labelenumi}{\arabic{enumi}.}
		% A(\Alph) a(\alph) I(\Roman) i(\roman) 1(\arabic)
		%设定全局标号series=example	%引用全局变量resume=example
		%[topsep=-0.3em,parsep=-0.3em,itemsep=-0.3em,partopsep=-0.3em]
		%可使用leftmargin调整列表环境左边的空白长度 [leftmargin=0em]
		\item
写出\((\hat{L}^{2},\hat{L}_z)\)表象中\(\hat{L_y}\)的矩阵表示。
	
	\item 
	证明\(\hat{L}_{y}^{3}=\hat{L}_{y}\)。
	
	\item 
	证明算符\(e^{-i\lambda\hat{L}_{y}}=1-i\hat{L}_{y}sin\lambda-(1-cos\lambda)\hat{L}_{y}^2\)。
	
	\item 
	求出\(e^{-i\lambda\hat{L}_{y}}\)在\((\hat{L}^{2},\hat{L}_y)\)表象中的矩阵表示。
	\end{enumerate}
\banswer{
	
}


\newpage	
	\item 
	一个质量为m的粒子处在一维无限深方势阱\(\left(-\frac{L}{2}<x<\frac{L}{2}\right)\)中,其中哈密顿量记为\(H_0\)。
	\begin{enumerate}
		%\renewcommand{\labelenumi}{\arabic{enumi}.}
		% A(\Alph) a(\alph) I(\Roman) i(\roman) 1(\arabic)
		%设定全局标号series=example	%引用全局变量resume=example
		%[topsep=-0.3em,parsep=-0.3em,itemsep=-0.3em,partopsep=-0.3em]
		%可使用leftmargin调整列表环境左边的空白长度 [leftmargin=0em]
		\item
求\(H_0\)的本征值和对应的波函数\(\psi(x)\)。
	
	\item 
	记受到微扰\(H^{\prime}(x)=\lambda \delta(x)(\lambda>0)\),求基态能量和基态波函数的一级修正。
	
	\item 
	严格求\(H=H_{0}+H'\)下的能级方程,并考虑\(\lambda \ll \frac{2 \hbar^{2}}{m L}\)的极限并与上述一级微扰论的基态能量比较。
\end{enumerate}	

\banswer{
	
}


\item 
由$ 3 $个电子组成的体系,设其自旋哈密顿量为\((\hbar=1)\)
$ \hat{H}=a(\hat{S_1^x}\hat{S_2^x}+\hat{S_1^y}\hat{S_2^y}+\hat{S_2^y}\hat{S_3^y}+\hat{S_2^x}\hat{S_3^x}+\hat{S_1^x}\hat{S_3^x}+\hat{S_1^y}\hat{S_3^y})-b\hat{S_T^z} $
其中\(\hat{S_i^\alpha}\)代表第$ i $个电子自旋算符的$\alpha $
分量,$\hat{S_T^z}$代表体系总自旋算符$\vec{S_T}=\vec{S_1}+\vec{S_2} + \vec{S_3}$的$ z $分量。
\begin{enumerate}
	%\renewcommand{\labelenumi}{\arabic{enumi}.}
	% A(\Alph) a(\alph) I(\Roman) i(\roman) 1(\arabic)
	%设定全局标号series=example	%引用全局变量resume=example
	%[topsep=-0.3em,parsep=-0.3em,itemsep=-0.3em,partopsep=-0.3em]
	%可使用leftmargin调整列表环境左边的空白长度 [leftmargin=0em]
	\item
求出总自旋平方算符\(\hat{S_T^2}\)的最大和最小本征值
	
	\item 
	计算以下两个对易式:\(\left[\hat{S_{T}^{2}}, \hat{H}\right]\)、\(\left[\hat{S_{T}^{z}}, \hat{H}\right]\)
	
	\item 
	求出\(\hat{H}\)的所有本征值。
\end{enumerate}

	
	\banswer{
		
	}
	
	
	
\end{enumerate}


