\bta{2006}

\begin{enumerate}
	%\renewcommand{\labelenumi}{\arabic{enumi}.}
	% A(\Alph) a(\alph) I(\Roman) i(\roman) 1(\arabic)
	%设定全局标号series=example	%引用全局变量resume=example
	%[topsep=-0.3em,parsep=-0.3em,itemsep=-0.3em,partopsep=-0.3em]
	%可使用leftmargin调整列表环境左边的空白长度 [leftmargin=0em]
	\item
一个质量为$\mu$的粒子被限制在$-a\le x\le a$内运动,$t=0$时处于基态。现势阱突然向两边对称地扩展一倍,即在$-2a\le x\le 2a$内运动。问:
\begin{enumerate}
	%\renewcommand{\labelenumi}{\arabic{enumi}.}
	% A(\Alph) a(\alph) I(\Roman) i(\roman) 1(\arabic)
	%设定全局标号series=example	%引用全局变量resume=example
	%[topsep=-0.3em,parsep=-0.3em,itemsep=-0.3em,partopsep=-0.3em]
	%可使用leftmargin调整列表环境左边的空白长度 [leftmargin=0em]
	\item
	$t=t_0(>0)$时粒子处于新系统基态的几率;
\item 
$t=t_0$时粒子能量的平均值。
\end{enumerate}

\banswer{
	
}


\item 
一维谐振子的哈密顿量为$\hat{H}=\frac{\hat{p}^2}{2m}+\frac{1}{2}\mu\omega^2x^2$。在坐标表象中,它的能量本征函数为:
$$\psi_n(x)=N_ne^{\frac{-a^2x^2}{2}}H_n(ax)\qquad a=\sqrt{\frac{\mu\omega}{\hbar}}$$
试在动量表象中求出它的能量本征值和相应的本征函数。

\banswer{
	
}


\item 
电子处于自旋$\vec{S}$在方向$n=(\sin\theta\cos\varphi,\sin\theta\sin\varphi,\cos\theta)$上投影$\vec{S}\cdot\vec{n}$的本征态,本征值为$\frac{\hbar}{2}$。
\begin{enumerate}
	%\renewcommand{\labelenumi}{\arabic{enumi}.}
	% A(\Alph) a(\alph) I(\Roman) i(\roman) 1(\arabic)
	%设定全局标号series=example	%引用全局变量resume=example
	%[topsep=-0.3em,parsep=-0.3em,itemsep=-0.3em,partopsep=-0.3em]
	%可使用leftmargin调整列表环境左边的空白长度 [leftmargin=0em]
	\item
求出相应的本征函数;

\item 
若在上面的态中,自旋的$x$分量和$y$分量有相等的均方差,请求出方向角$\theta,\varphi$。
\end{enumerate}


\banswer{
	
}

\newpage
\item 
自旋$\frac{1}{2}$的粒子处于磁场$\vec{B}$中,该粒子绕磁场进动的角频率记为$\omega=-r\vec{B}$。设$t=0$时粒子处于自旋朝下态$|\psi(0)\rangle=|-\rangle$,求$t$时刻粒子仍处于该态的几率。

\banswer{
	
}


\item 
在谐振子的哈密顿量$\hat{H}_0=\frac{1}{2\mu}\hat{p}^2+\frac{1}{2}\mu\omega^2x^2$上加上$x^3$的微扰项$H'=\lambda x^3$,求能量的二级修正。

\banswer{
	
}


\item 
有一量子力学体系,哈密顿量$\hat{H}$的本征值与本征矢分别为$E_n$与$|n\rangle $,
$\hat{H}|n\rangle =E_n|n\rangle $。设$\hat{F}$为任一算符$\hat{F}=\hat{F}(x,\hat{p})$,试证明:
$$\langle  k|[F^+,[H,F]]|k\rangle =\sum_n(E_n-E_k)(|\langle  n|\hat{F}|k\rangle |^2+|\langle  k|\hat{F}|n\rangle |^2)$$

\banswer{
	
}

	
\end{enumerate}

