\bta{2006甲A}

\begin{enumerate}
	%\renewcommand{\labelenumi}{\arabic{enumi}.}
	% A(\Alph) a(\alph) I(\Roman) i(\roman) 1(\arabic)
	%设定全局标号series=example	%引用全局变量resume=example
	%[topsep=-0.3em,parsep=-0.3em,itemsep=-0.3em,partopsep=-0.3em]
	%可使用leftmargin调整列表环境左边的空白长度 [leftmargin=0em]
	\item
($30'$)两个线性算符$\hat{A}$和$\hat{B}$满足下列关系:$\hat{A}^2=0,\quad \hat{A}\hat{A}^\dagger+\hat{A}^\dagger \hat{A}=1,\quad \hat{B}=\hat{A}^\dagger\hat{A}$。
\begin{enumerate}
	%\renewcommand{\labelenumi}{\arabic{enumi}.}
	% A(\Alph) a(\alph) I(\Roman) i(\roman) 1(\arabic)
	%设定全局标号series=example	%引用全局变量resume=example
	%[topsep=-0.3em,parsep=-0.3em,itemsep=-0.3em,partopsep=-0.3em]
	%可使用leftmargin调整列表环境左边的空白长度 [leftmargin=0em]
	\item
求证$\hat{B}^2=\hat{B}$;

\item 
求在$\hat{B}$表象中$\hat{A}$和$\hat{B}$的表达式。
	
\end{enumerate}


\banswer{
	
}


\item 
($30'$)粒子在势场$V(x)=A|x|^n\quad(-\infty <x<\infty,A>0)$中运动,试用不确定度关系估算基态能量。

\banswer{
	
}


\item 
($30'$)设体系的哈密顿量$\hat{H}$从依赖于某一参量$\lambda$,又设体系处于某一束缚定态,其能量和本征函数分别记为$E_n$和$\psi_n(r)$。
\begin{enumerate}
	%\renewcommand{\labelenumi}{\arabic{enumi}.}
	% A(\Alph) a(\alph) I(\Roman) i(\roman) 1(\arabic)
	%设定全局标号series=example	%引用全局变量resume=example
	%[topsep=-0.3em,parsep=-0.3em,itemsep=-0.3em,partopsep=-0.3em]
	%可使用leftmargin调整列表环境左边的空白长度 [leftmargin=0em]
	\item
证明费曼——海尔曼定理:
$$\frac{\partial E_n}{\partial \lambda}=\int \psi^*_n(\vec{r})\frac{\partial \hat{H}}{\partial \lambda}\psi_n(\vec{r})d\vec{r}$$

\item 
利用费曼——海尔曼定理,求氢原子各束缚态的平均动能。\\
(提示:氢原子能级公式$E_n=-\frac{\mu e^4}{2\hbar^2}\cdot\frac{1}{n^2}$)
	
\end{enumerate}


\banswer{
	
}

\newpage
\item 
($30'$)粒子在二维无限深方势阱中运动,$V=\begin{cases}0,&0<x<a,0<y<a\\ \infty,&\text{其他}\end{cases}$。加上微扰$H'=\lambda xy$后,求基态和第一激发态能级的一级微扰修正。

\banswer{
	
}


\item 
($30'$)设粒子所处的外场均匀但与时间有关。即$V=V(t)$,与坐标$\vec{r}$无关。试将体系的含时薛定鄂方程分离变量,求方程解$\psi(\vec{r},t)$的一般形式,并取$V(t)=V_0\cos(\omega t)$,以一维情况为例说明$V(t)$的影响是什么。

\banswer{
	
}



\end{enumerate}



