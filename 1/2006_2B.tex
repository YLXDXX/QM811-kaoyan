

\bta{2006乙B}

\begin{enumerate}
	%\renewcommand{\labelenumi}{\arabic{enumi}.}
	% A(\Alph) a(\alph) I(\Roman) i(\roman) 1(\arabic)
	%设定全局标号series=example	%引用全局变量resume=example
	%[topsep=-0.3em,parsep=-0.3em,itemsep=-0.3em,partopsep=-0.3em]
	%可使用leftmargin调整列表环境左边的空白长度 [leftmargin=0em]
	\item
($30'$)粒子以能量$E$入射方势垒,$V(x)=\begin{cases}V_0>0,&0\le x\le a\\0,&x<0,x>0\end{cases}$。设能量$E<V_0$,求透射系数$T$。


\banswer{
	
}


\item 
($30'$)粒子在一维对称无限深方势阱$(-\frac{a}{2}\le x\le \frac{a}{2})$中运动。设$t=0$时粒子所处状态为$\psi(x,t=0)=\frac{1}{\sqrt2}[\varphi_1(x)+\varphi_2(x)]$,其中$\varphi_n(x)$为系统第$n$个能量本征态。求$t>0$时的以下量:
\begin{enumerate}
	%\renewcommand{\labelenumi}{\arabic{enumi}.}
	% A(\Alph) a(\alph) I(\Roman) i(\roman) 1(\arabic)
	%设定全局标号series=example	%引用全局变量resume=example
	%[topsep=-0.3em,parsep=-0.3em,itemsep=-0.3em,partopsep=-0.3em]
	%可使用leftmargin调整列表环境左边的空白长度 [leftmargin=0em]
	\item
概率密度$|\psi(x,t)|^2$;

\item 
能量的可取值及相应的概率。
	
\end{enumerate}


\banswer{
	
}


\item 
($30'$)设氢原子所处状态为$\psi(r,\theta,\phi,S_z)=\begin{cases}\frac{1}{2}R_{21}(r)Y_{11}(\theta,\phi)\\-\frac{\sqrt3}{2}R_{21}(r)Y_{10}(\theta,\phi)\end{cases}$。
\begin{enumerate}
	%\renewcommand{\labelenumi}{\arabic{enumi}.}
	% A(\Alph) a(\alph) I(\Roman) i(\roman) 1(\arabic)
	%设定全局标号series=example	%引用全局变量resume=example
	%[topsep=-0.3em,parsep=-0.3em,itemsep=-0.3em,partopsep=-0.3em]
	%可使用leftmargin调整列表环境左边的空白长度 [leftmargin=0em]
	\item
求轨道角动量$z$分量$\hat{L}_z$和自旋角动量$z$分量$\hat{S}_z$的平均值;

\item 
求总磁矩$\vec{M}=-\frac{e}{2\mu}\vec{L}-\frac{e}{\mu}\vec{S}$的$z$分量的平均值。
	
\end{enumerate}

\banswer{
	
}

\newpage
\item 
对于一维谐振子的基态,求坐标和动量的不确定度的乘积$\Delta x\cdot\Delta p$。

\banswer{
	
}


\item 
($30'$)两个自旋为$\frac{1}{2}$非全同粒子,自旋间相互作用为$\hat{H}=J\vec{s}_1\cdot\vec{s}_2$,其中$\vec{s}_1$和$\vec{s}_2$分别为粒子1和粒子2的自旋算符。设$t=0$时粒子1的自旋沿$z$轴正方向,粒子2的自旋沿$z$轴负方向。求$t>0$时,测到粒子2的自旋仍处于$z$轴负方向的概率。


\banswer{
	
}


	
\end{enumerate}
