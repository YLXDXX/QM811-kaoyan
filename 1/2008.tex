
\bta{2008}

\begin{enumerate}
	%\renewcommand{\labelenumi}{\arabic{enumi}.}
	% A(\Alph) a(\alph) I(\Roman) i(\roman) 1(\arabic)
	%设定全局标号series=example	%引用全局变量resume=example
	%[topsep=-0.3em,parsep=-0.3em,itemsep=-0.3em,partopsep=-0.3em]
	%可使用leftmargin调整列表环境左边的空白长度 [leftmargin=0em]
	\item
($30'$)写成氢原子的束缚态能级,所有量子数以及取值范围,求出其简并度。


\banswer{
	
}


\item 
($30'$)一个粒子质量为$\mu$,在一势能环中运动,势能
$$V=\begin{cases}0,&0<\varphi<\varphi_0\\ \infty,&other\end{cases}$$
求粒子运动的本征值和本征函数。


\banswer{
	
}


\item 
($30'$)求在$H=\begin{pmatrix}1&\lambda &0\\ \lambda &3&0 \\ 0&0& \lambda-2\end{pmatrix}$中粒子的本征值,设$\lambda \ll 1$,利用微扰求其本征值(精确到二级近似)并与精确求解相比较。


\banswer{
	
}




\newpage
\item 
($30'$)两个自旋为$\frac{1}{2}\hbar$的粒子,两个粒子分别为$\chi_1=\begin{pmatrix}1\\0\end{pmatrix}$,$\chi_1=\begin{pmatrix}\cos\theta e^{-i\omega t}\\ \sin\theta e^{-i\omega t}\end{pmatrix}$,求系统处于单态和三态的概率。

\banswer{
	
}


\item 
($30'$)处在一维谐振子势基态的粒子受到微扰$H'=\lambda x\delta(t_0)$作用,求跃迁到其他各激发态的总概率和仍处在基态的概率。已知:
$xH_n=\frac{1}{\sqrt a}\left[\sqrt{\frac{n}{2}}H_{n-1}+\sqrt{\frac{n+1}{2}}H_{n+1}\right]$。

\banswer{
	
}



\end{enumerate}

