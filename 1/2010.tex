\bta{2010}


\begin{enumerate}
	%\renewcommand{\labelenumi}{\arabic{enumi}.}
	% A(\Alph) a(\alph) I(\Roman) i(\roman) 1(\arabic)
	%设定全局标号series=example	%引用全局变量resume=example
	%[topsep=-0.3em,parsep=-0.3em,itemsep=-0.3em,partopsep=-0.3em]
	%可使用leftmargin调整列表环境左边的空白长度 [leftmargin=0em]
	\item


\begin{enumerate}
	%\renewcommand{\labelenumi}{\arabic{enumi}.}
	% A(\Alph) a(\alph) I(\Roman) i(\roman) 1(\arabic)
	%设定全局标号series=example	%引用全局变量resume=example
	%[topsep=-0.3em,parsep=-0.3em,itemsep=-0.3em,partopsep=-0.3em]
	%可使用leftmargin调整列表环境左边的空白长度 [leftmargin=0em]
	\item
设$\hat{A},\hat{B}$与Pauli算符对易,证明:
$(\vec{\sigma}\cdot\vec{A})(\vec{\sigma}\cdot\vec{B})=\vec{A}\cdot\vec{B}+i\vec{\sigma}\cdot(\vec{A}\times\vec{B})$

\item 
试将$(\hat{I}+\hat{\sigma}_x+i\hat{\sigma}_y)^{\frac{1}{2}}$表示成$\hat{I},\hat{\sigma}_x,\hat{\sigma}_y,\hat{\sigma_z}$的线性叠加,$\hat{I}$为单位算符。

\end{enumerate}
\banswer{
\begin{enumerate}
	%\renewcommand{\labelenumi}{\arabic{enumi}.}
	% A(\Alph) a(\alph) I(\Roman) i(\roman) 1(\arabic)
	%设定全局标号series=example	%引用全局变量resume=example
	%[topsep=-0.3em,parsep=-0.3em,itemsep=-0.3em,partopsep=-0.3em]
	%可使用leftmargin调整列表环境左边的空白长度 [leftmargin=0em]
	\item
	证明略
	\item 
	$\sqrt{I+\sigma_{x}+i \sigma_{y}}=I+\frac{1}{2}\left(\sigma_{x}+i \sigma_{y}\right)$
	或者
	$\sqrt{I+\sigma_{x}+i \sigma_{y}}=-I-\frac{1}{2}\left(\sigma_{x}+i \sigma_{y}\right)$
	
\end{enumerate}
\begin{note}
	表示方法不唯一,这是由于根号的多值性决定的
\end{note}

}




\item 
设一维谐振子的初态为$\psi(x,0)=\cos\frac{\theta}{2}\varphi_0(x)+\sin\frac{\theta}{2}\varphi_1(x)$,
即基态与第一激发态的叠加,其中$\theta$为实参数:
\begin{enumerate}
	%\renewcommand{\labelenumi}{\arabic{enumi}.}
	% A(\Alph) a(\alph) I(\Roman) i(\roman) 1(\arabic)
	%设定全局标号series=example	%引用全局变量resume=example
	%[topsep=-0.3em,parsep=-0.3em,itemsep=-0.3em,partopsep=-0.3em]
	%可使用leftmargin调整列表环境左边的空白长度 [leftmargin=0em]
	\item
求$t$时刻的波函数$\psi(x,t)$;

\item 
求$t$时刻粒子处于基态及第一激发态的概率;

\item 
求$t$时刻粒子的势能算符$\hat{V}=\frac{m\omega^2}{2}x^2$的平均值;

\item 
求演化成$-\psi(x,0)$所需的最短时间$t_{min}$。
\end{enumerate}
\banswer{
\begin{enumerate}
	%\renewcommand{\labelenumi}{\arabic{enumi}.}
	% A(\Alph) a(\alph) I(\Roman) i(\roman) 1(\arabic)
	%设定全局标号series=example	%引用全局变量resume=example
	%[topsep=-0.3em,parsep=-0.3em,itemsep=-0.3em,partopsep=-0.3em]
	%可使用leftmargin调整列表环境左边的空白长度 [leftmargin=0em]
	\item
	$\psi(x, t)=\cos \frac{\theta}{2}|0\rangle e^{-\frac{i \omega t}{2}}+\sin \frac{\theta}{2}|1\rangle e^{-\frac{3 i \omega t}{2}}$
	\item 
	$P_{0}=\cos ^{2} \frac{\theta}{2}$
	 \quad $P_{1}=\sin ^{2} \frac{\theta}{2}$
	\item 
	$\langle V\rangle=\frac{m \omega^{2}}{2}\left\langle x^{2}\right\rangle=\frac{\hbar \omega}{4} \cos ^{2} \frac{\theta}{2}+\frac{3 \hbar \omega}{4} \sin ^{2} \frac{\theta}{2}$
	\item 
	$ t_{\min}=\frac{2\pi}{\omega} $
\end{enumerate}
}



\item 
设基态氢原子处于弱电场中,微扰哈密顿量为:
$$
\hat{H} ^{\prime} =\begin{cases}0,&t\le0\\
\lambda ze^{-\frac{t}{T}},&t>0
\end{cases}
$$
其中$  \lambda,T $为常数。
\begin{enumerate}
	%\renewcommand{\labelenumi}{\arabic{enumi}.}
	% A(\Alph) a(\alph) I(\Roman) i(\roman) 1(\arabic)
	%设定全局标号series=example	%引用全局变量resume=example
	%[topsep=-0.3em,parsep=-0.3em,itemsep=-0.3em,partopsep=-0.3em]
	%可使用leftmargin调整列表环境左边的空白长度 [leftmargin=0em]
	\item
求很长时间后$(t\gg T)$电子跃迁到激发态的概率。
基态和激发态波函数为:
$$
\begin{aligned}
	&\psi_{100}(\vec{r})=R_{10}(r)Y_{00}(\theta,\varphi)=\frac{2}{a^{\frac{3}{2}}}e^{-\frac{r}{a}} \cdot \frac{1}{\sqrt{4\pi}} 
	\\ 
	&\psi_{210}(\vec{r})=R_{21}(r)Y_{10}(\theta,\varphi)=\frac{1}{(2a)^{\frac{3}{2}}}\frac{r}{\sqrt3a}e^{-\frac{r}{2a}} \cdot \sqrt{\frac{3}{4\pi}}\cos\theta 
\end{aligned}
$$
其中$  a  $为玻尔半径。
\item 
基态电子跃迁到下列哪个激发态的概率等于零?简述理由。
\fourchoices
{$\psi_{200}$}
{$\psi_{211}$}
{$\psi_{21,-1}$}
{$\psi_{210}$}


	
\end{enumerate}

\banswer{
采用球坐标: $\hat{H}^{\prime}=\lambda z e^{-\frac{t}{T}}=\lambda e^{-\frac{t}{T}} r \cos \theta$,注意到
\[ \cos \theta Y_{l m}=\sqrt{\frac{(l+1)^{2}-m^{2}}{(2 l+1)(2 l+3)}} \mathrm{Y}_{l+1, m}+\sqrt{\frac{l^{2}-m^{2}}{(2 l-1)(2 l+1)}} \mathrm{Y}_{l-1, m} \]
\[ 
a_{m n}=\frac{1}{i \hbar} \int_{0}^{t} H_{m n}^{\prime} e^{i \omega_{m n} t^{\prime}} d t^{\prime}
\]
\begin{enumerate}
	%\renewcommand{\labelenumi}{\arabic{enumi}.}
	% A(\Alph) a(\alph) I(\Roman) i(\roman) 1(\arabic)
	%设定全局标号series=example	%引用全局变量resume=example
	%[topsep=-0.3em,parsep=-0.3em,itemsep=-0.3em,partopsep=-0.3em]
	%可使用leftmargin调整列表环境左边的空白长度 [leftmargin=0em]
	\item
	\[ 
	H_{21}^{\prime}=\left\langle\psi_{210}\left|\hat{H}^{\prime}\right| \psi_{100}\right\rangle=4\sqrt{2} \cdot \left(\frac{2}{3}\right)^{5} \cdot a \cdot e^{-\frac{t}{T}}
	 \]
	 \[ 
	 P=\ 2^{5} \left( \frac{ 2 }{ 3 } \right)^{10} \lambda^{2}  \cdot \frac{a^{2}}{\hbar^{2}} \cdot \frac{T^{2}}{1+\omega_{21}^{2}T^{2}}
	  \]
	\item 
	根据$ \cos \theta Y_{l m} $的性质,此时的选择定则为$ \Delta m =0 , \Delta l=\pm 1 $,
	所以基态电子跃迁到激发态 $ \psi_{200}, \psi_{211}, \psi_{21,-1}$ 的概率均等于零。
\end{enumerate}

	
}


\newpage

\item 
两个质量为$m$的粒子处于一个边长为$a>b>c$的不可穿透的长盒子中,求下列条件下,该体系能量最低态的波函数(只写出空间部分)及对应能量。
\begin{enumerate}
	%\renewcommand{\labelenumi}{\arabic{enumi}.}
	% A(\Alph) a(\alph) I(\Roman) i(\roman) 1(\arabic)
	%设定全局标号series=example	%引用全局变量resume=example
	%[topsep=-0.3em,parsep=-0.3em,itemsep=-0.3em,partopsep=-0.3em]
	%可使用leftmargin调整列表环境左边的空白长度 [leftmargin=0em]
	\item
非全同粒子。
\item 
零自旋全同粒子。
\item 
自旋为$\frac{1}{2}$的全同粒子。
\end{enumerate}
\banswer{
一个粒子处于长盒子中的波函数及相应能量为
$$
\phi(x, y, z)=\frac{2 \sqrt{2}}{\sqrt{a b c}} \sin \frac{n_{1} \pi x}{a} \sin \frac{n_{2} \pi x}{b} \sin \frac{n_{3} \pi z}{c} \quad e_{n}=\frac{\pi^{2} \hbar^{2}}{2 m}\left(\frac{n_{1}^{2}}{a^{2}}+\frac{n_{2}^{2}}{b^{2}}+\frac{n_{3}^{2}}{c^{2}}\right)
$$
\begin{enumerate}
	%\renewcommand{\labelenumi}{\arabic{enumi}.}
	% A(\Alph) a(\alph) I(\Roman) i(\roman) 1(\arabic)
	%设定全局标号series=example	%引用全局变量resume=example
	%[topsep=-0.3em,parsep=-0.3em,itemsep=-0.3em,partopsep=-0.3em]
	%可使用leftmargin调整列表环境左边的空白长度 [leftmargin=0em]
	\item
	非全同粒子基态:
	$$\psi_{1}\left(r_{1}, r_{2}\right)=\phi_{1}\left(r_{1}\right) \phi_{1}\left(r_{2}\right)=\frac{8}{a b c} \sin \frac{\pi x_{1}}{a} \sin \frac{\pi y_{1}}{b} \sin \frac{\pi z_{1}}{c} \sin \frac{\pi x_{2}}{a} \sin \frac{\pi y_{2}}{b} \sin \frac{\pi z_{2}}{c}$$
	$$
	E=e_{1}+e_{1}=\frac{\pi^{2} \hbar^{2}}{m}\left(\frac{1}{a^{2}}+\frac{1}{b^{2}}+\frac{1}{c^{2}}\right)
	$$
	\item 
	零自旋全同粒子基态,零自旋粒子可以处于同一态,与非全同粒子基态相同:
	$$
	\psi_{1}\left(r_{1}, r_{2}\right)=\frac{1}{2}\left[\phi_{1}\left(r_{1}\right) \phi_{1}\left(r_{2}\right)+\phi_{1}\left(r_{2}\right) \phi_{1}\left(r_{1}\right)\right]=\phi_{1}\left(r_{1}\right) \phi_{1}\left(r_{2}\right)
	$$
	\item 
	自旋为 $\frac{1}{2}$ 的全同粒子需满足泡利不相容原理, 由于 $a>b>c$,故第一激发态为
	$$
	\phi_{2}(r)=\frac{2 \sqrt{2}}{\sqrt{a b c}} \sin \frac{2 \pi x}{a} \sin \frac{\pi y}{b} \sin \frac{\pi z}{c}
	$$
	$$
	\psi_{1}\left(r_{1}, r_{2}\right)=\frac{1}{\sqrt{2}}\left[\phi_{1}\left(r_{1}\right) \phi_{2}\left(r_{2}\right)-\phi_{2}\left(r_{1}\right) \phi_{1}\left(r_{2}\right)\right]
 \quad 
E=\frac{\pi^{2} \hbar^{2}}{m}\left(\frac{5}{2} \cdot \frac{1}{a^{2}}+\frac{1}{b^{2}}+\frac{1}{c^{2}}\right)
$$
\end{enumerate}
\begin{note}
此题要求只写出空间部分,一般情况下还有自旋部分。当盒子是方盒子即 $a=b=c$ 时,第二激发态$\phi_{2}$ 存在简并,三重,对应的波函数有$ 3 $个。另外波函数的形式随着坐标原点的选择不同而不同(也可将坐标原点选在中心)。
\end{note}

}



\item 
粒子在宽度为$ 2a $的一维无限深势阱中运动
\[ 
V(x)=
\left\{
\begin{array}{lr}
	0,&0<x<2a\\
	\infty, & x\leq 0,x\geq2a
\end{array}
\right.
 \]
设该体系受到$\hat{H} ^{\prime} =\lambda\delta(x-a)$的微扰作用。
\begin{enumerate}
	%\renewcommand{\labelenumi}{\arabic{enumi}.}
	% A(\Alph) a(\alph) I(\Roman) i(\roman) 1(\arabic)
	%设定全局标号series=example	%引用全局变量resume=example
	%[topsep=-0.3em,parsep=-0.3em,itemsep=-0.3em,partopsep=-0.3em]
	%可使用leftmargin调整列表环境左边的空白长度 [leftmargin=0em]
	\item
利用微扰理论,求第$n$能级精确到二级的近似表达式。

\item 
指出所得结果的适用条件。
\end{enumerate}

\banswer{
无微扰时波函数和能级为
\[
\phi_{n}^{(0)}=\sqrt{\frac{1}{a}} \sin \frac{n \pi x}{2 a} \quad E_{n}^{(0)}=\frac{n^{2} \pi^{2} \hbar^{2}}{8 m a^{2}}
\]
\begin{enumerate}
	%\renewcommand{\labelenumi}{\arabic{enumi}.}
	% A(\Alph) a(\alph) I(\Roman) i(\roman) 1(\arabic)
	%设定全局标号series=example	%引用全局变量resume=example
	%[topsep=-0.3em,parsep=-0.3em,itemsep=-0.3em,partopsep=-0.3em]
	%可使用leftmargin调整列表环境左边的空白长度 [leftmargin=0em]
	\item
第$n$能级精确到二级的近似表达式为:
\[ 
	\begin{aligned}
	E_{n}&=E_{n}^{(0)}+E_{n}^{(1)}+E_{n}^{(2)}\\
	&=\frac{n^{2} \pi^{2} \hbar^{2}}{8 m a^{2}}+\left\langle\psi_{n}^{0}\left|H^{\prime}\right| \psi_{n}^{0}\right\rangle+\sum_{k} {}^{\prime} \frac{\left|H_{k n}^{\prime}\right|^{2}}{E_{n}^{0}-E_{k}^{0}}
\end{aligned}
 \]
 当$ n $为奇数的时候
 \[ 
 E_{n}=\frac{n^{2} \pi^{2} \hbar^{2}}{8 m a^{2}} + 0+0
  \]
  当$ n $为偶数的时候
 \[ 
 E_{n}=\frac{n^{2} \pi^{2} \hbar^{2}}{8 m a^{2}} + \frac{\lambda}{a}+\frac{8 m \lambda^{2}}{\pi^{2} \hbar^{2}}\left[\frac{1}{n^{2}-1}+\frac{1}{n^{2}-n^{3}}+\frac{1}{n^{2}-5^{2}}+\cdots\right]
  \]
  	其中:
  \begin{equation*}\label{key}
  	\begin{aligned}
  		H_{k n}^{\prime}&=\left\langle\phi_{k}^{0}\left|H^{\prime}\right| \phi_{n}^{0}\right\rangle=\frac{\lambda}{a} \int_{0}^{2 a} \delta(x-a) \sin \frac{k \pi x}{2 a} \sin \frac{n \pi x}{2 a} d x
  		\\
  		&=\frac{\lambda}{a} \sin \frac{k \pi}{2} \sin \frac{n \pi}{2}=\left\{\begin{array}{lc}
  			n \text { 为偶数 }, k \neq n, & 0 \\
  			n \text { 为奇数 }, k \text { 为奇数, } k \neq n, & \frac{\lambda}{a} \\
  			n \text { 为奇数 }, k \text { 为偶数, } k \neq n, & 0
  		\end{array}\right.
  	\end{aligned}
  \end{equation*}
\item 
适用条件:$\left|\frac{H_{k n}^{\prime}}{E_{n}^{0}-E_{k}^{0}}\right| \ll 1$ 即: $\left|\frac{\frac{\lambda}{a}}{E_{n}^{0}-E_{k}^{0}}\right| \ll 1$,  $\lambda$ 很小,能级无简并或无近似简并。	
\end{enumerate}


}



\end{enumerate}


