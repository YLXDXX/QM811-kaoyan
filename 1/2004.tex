

\bta{2004}

\begin{enumerate}
	%\renewcommand{\labelenumi}{\arabic{enumi}.}
	% A(\Alph) a(\alph) I(\Roman) i(\roman) 1(\arabic)
	%设定全局标号series=example	%引用全局变量resume=example
	%[topsep=-0.3em,parsep=-0.3em,itemsep=-0.3em,partopsep=-0.3em]
	%可使用leftmargin调整列表环境左边的空白长度 [leftmargin=0em]
	\item
($30'$)粒子在一维无限深方势阱$V(x)$中运动,$V(x)=\begin{cases}\infty,&|x|>a\\0,&|x|<a\end{cases}$处于状态$\psi=\phi_1+\phi_3+2\phi_4$。这里$\phi_n,n=1,2,3,\cdots$是系统归一化的能量本征态。请问:
\begin{enumerate}
	%\renewcommand{\labelenumi}{\arabic{enumi}.}
	% A(\Alph) a(\alph) I(\Roman) i(\roman) 1(\arabic)
	%设定全局标号series=example	%引用全局变量resume=example
	%[topsep=-0.3em,parsep=-0.3em,itemsep=-0.3em,partopsep=-0.3em]
	%可使用leftmargin调整列表环境左边的空白长度 [leftmargin=0em]
	\item
粒子具有基态能量$E_1$几率;
\item 
粒子的平均能量(用基态能量$E_1$的倍数表示);
\item 
态$\phi_4$中的节点数(在节点处,找到粒子的几率密度为零);
\item 
态$\phi_3$的宇称。

\end{enumerate}

\banswer{
	
}


\item 
考虑一维体系$\hat{H}=\frac{p^2}{2\mu}+V(x),\quad V(x)=V_0x^\lambda,\quad V_0>0,\quad \lambda=2,4,6\cdots$。设$\hat{H}$的本征波函数为$\psi_n$。
\begin{enumerate}
	%\renewcommand{\labelenumi}{\arabic{enumi}.}
	% A(\Alph) a(\alph) I(\Roman) i(\roman) 1(\arabic)
	%设定全局标号series=example	%引用全局变量resume=example
	%[topsep=-0.3em,parsep=-0.3em,itemsep=-0.3em,partopsep=-0.3em]
	%可使用leftmargin调整列表环境左边的空白长度 [leftmargin=0em]
	\item
证明动量在态$\psi_n$中的平均值为零;

\item 
求在态$\psi_n$中的动能平均值和势能平均值之间的关系。

	
\end{enumerate}

\banswer{
	
}



\item 
设归一化的状态波函数$|\psi\rangle$满足薛定鄂方程$i\hbar\frac{\partial}{\partial t}|\psi\rangle=\hat{H}|\psi\rangle$,定义密度算符(矩阵)为$\rho=|\psi\rangle\langle\psi|$。
\begin{enumerate}
	%\renewcommand{\labelenumi}{\arabic{enumi}.}
	% A(\Alph) a(\alph) I(\Roman) i(\roman) 1(\arabic)
	%设定全局标号series=example	%引用全局变量resume=example
	%[topsep=-0.3em,parsep=-0.3em,itemsep=-0.3em,partopsep=-0.3em]
	%可使用leftmargin调整列表环境左边的空白长度 [leftmargin=0em]
	\item
证明任意力学量$\hat{F}$在态$|\psi\rangle$中的平均值可表示为$Tr(\rho F)$;

\item 
求出$\rho$的本征值;

\item 
导出随时间演化的方程。
\end{enumerate}

\banswer{
	
}


\newpage
\item 
质量为$\mu$的粒子在三维各向同性谐振子势为$V(r)=\frac{kr^2}{2}=\frac{k(x^2+y^2+z^2)}{2}$中运动。求:
\begin{enumerate}
	%\renewcommand{\labelenumi}{\arabic{enumi}.}
	% A(\Alph) a(\alph) I(\Roman) i(\roman) 1(\arabic)
	%设定全局标号series=example	%引用全局变量resume=example
	%[topsep=-0.3em,parsep=-0.3em,itemsep=-0.3em,partopsep=-0.3em]
	%可使用leftmargin调整列表环境左边的空白长度 [leftmargin=0em]
	\item
第二激发态的能量;
\item 
第一激发态的简并度;
\item 
在基态中的不确定量$\Delta r\cdot\Delta p$,这里$\Delta r$是位置矢量的均方差根。定
义类似。
\end{enumerate}


\banswer{
	
}


\item 
两个自旋都是 $1 / 2$ 的粒子 $\mathbf{1}$ 和 $\mathbf{2}$ 组成的系统, 处于由波函数 $|\psi\rangle=a|0\rangle_{1}|1\rangle_{2}+b|1\rangle_{1}|0\rangle_{2}$ 描写的状态, 其中 $|0\rangle$ 表示自旋朝下(沿 $-z$ 方向 $),|1\rangle$ 表示自旋朝上。当数 $\mathbf{a}$ 和 $\mathbf{b}$ 都不 为 $\mathbf{0}$ 时, 此态不能表示成两个单个粒子状态的直接乘积形式$ |\rangle_{1}|\rangle_{2}$ 时称为纠湹态。试 求在上面的纠缠态。试求在上面的纠湹态中
\begin{enumerate}
	%\renewcommand{\labelenumi}{\arabic{enumi}.}
	% A(\Alph) a(\alph) I(\Roman) i(\roman) 1(\arabic)
	%设定全局标号series=example	%引用全局变量resume=example
	%[topsep=-0.3em,parsep=-0.3em,itemsep=-0.3em,partopsep=-0.3em]
	%可使用leftmargin调整列表环境左边的空白长度 [leftmargin=0em]
	\item
两个粒子的自旋互相平行的几率;
\item
两个粒子的自旋互相反平行的几率;
\item
此系统处于总自旋为 $\mathbf{0}$ 的几率;	
	
\item 
测量得到粒子 $ 1 $ 自旋朝上的几率多大; 发现粒子 $ 1 $ 自旋朝上时,粒子 $ 2 $ 处于什么 状态?


	
	
\end{enumerate}


\banswer{
	
}

\item 
考虑到自旋轨道耦合的氢原子,其哈密顿量为 $\hat{H}=-\frac{\hbar^{2}}{2 \mu} \nabla^{2}+c(r)+\xi(r) \hat{\vec{L}} \cdot \hat{\vec{S}}$。
\begin{enumerate}
	%\renewcommand{\labelenumi}{\arabic{enumi}.}
	% A(\Alph) a(\alph) I(\Roman) i(\roman) 1(\arabic)
	%设定全局标号series=example	%引用全局变量resume=example
	%[topsep=-0.3em,parsep=-0.3em,itemsep=-0.3em,partopsep=-0.3em]
	%可使用leftmargin调整列表环境左边的空白长度 [leftmargin=0em]
	\item
	证明轨道角动量 $\hat{\vec{L}}$ 和 $\hat{\vec{S}}$ 不是此系统的守恒量, 而总角动量 $\hat{\vec{J}}=\hat{\vec{L}}+\hat{\vec{S}}$ 是守恒量。
	\item
	若自旋一轨道相互作用 $\xi(r) \hat{\vec{L}} \cdot \hat{\vec{S}}$ 可当作微扰,计算此系统基态能量的一级修正。 \\
	($\hat{H}_{0}=-\frac{\hbar^{2}}{2 \mu} \nabla^{2}+V(r)$ 的本征能量为 $E_{n}^{(0)}$, 本征函数: $\psi_{n l m s}=R_{n l}(r) Y_{l m}(\theta, \varphi) \chi_{s}, \chi_{s}$为自旋波函数)
	
	
	
\end{enumerate}

\banswer{
	
}






\end{enumerate}

