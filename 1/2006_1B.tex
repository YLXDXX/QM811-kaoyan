
\bta{2006甲B}

\begin{enumerate}
	%\renewcommand{\labelenumi}{\arabic{enumi}.}
	% A(\Alph) a(\alph) I(\Roman) i(\roman) 1(\arabic)
	%设定全局标号series=example	%引用全局变量resume=example
	%[topsep=-0.3em,parsep=-0.3em,itemsep=-0.3em,partopsep=-0.3em]
	%可使用leftmargin调整列表环境左边的空白长度 [leftmargin=0em]
	\item
($30'$)已知谐振子处于第$n$个定态中,试导出算符$\hat{x},\hat{p},(\hat{x})^2,(\hat{p})^2$的平均值及不确定度$\Delta x,\Delta p$,并求出$\Delta x\cdot\Delta p$值。


\banswer{
	
}


\item 
($30'$)设$\hat{\mu}$为幺正算符,若存在两个厄米算符$\hat{A}$和$\hat{B}$使
$\hat{\mu}=\hat{A}+i\hat{B}$。试证:
\begin{enumerate}
	%\renewcommand{\labelenumi}{\arabic{enumi}.}
	% A(\Alph) a(\alph) I(\Roman) i(\roman) 1(\arabic)
	%设定全局标号series=example	%引用全局变量resume=example
	%[topsep=-0.3em,parsep=-0.3em,itemsep=-0.3em,partopsep=-0.3em]
	%可使用leftmargin调整列表环境左边的空白长度 [leftmargin=0em]
	\item
	$\hat{A}^2+\hat{B}^2=1$,且$[\hat{A},\hat{B}]=0$;

\item 
进一步再证明$\hat{\mu}$可表示成$\hat{\mu}=e^{i\hat{H}}$,$\hat{H}$为厄米算符。
\end{enumerate}

\banswer{
	
}


\item 
($30'$)一个质量为$m$的粒子被限制在$0\le x\le a$的一维无穷深势阱中。初始时刻其归一化波函数为$\psi(x,0)=\sqrt{\frac{8}{5a}}(1+\cos\frac{\pi x}{a})\sin\frac{\pi x}{a}$,求:
\begin{enumerate}
	%\renewcommand{\labelenumi}{\arabic{enumi}.}
	% A(\Alph) a(\alph) I(\Roman) i(\roman) 1(\arabic)
	%设定全局标号series=example	%引用全局变量resume=example
	%[topsep=-0.3em,parsep=-0.3em,itemsep=-0.3em,partopsep=-0.3em]
	%可使用leftmargin调整列表环境左边的空白长度 [leftmargin=0em]
	\item
$t>0$时粒子的状态波函数;

\item 
在$t=0$与$t>0$时在势阱的左半部发现粒子的概率是多少?
\end{enumerate}


\banswer{
	
}


\newpage
\item 
($30'$)粒子在一维无限深方势阱中运动,受到微扰$H'=\frac{V_0}{a}(a-|2x-a|)$的作用。求第$n$个能级的一级近似,并分析所的结果的适用条件。

\banswer{
	
}


\item 
一个质量为$m$ 的粒子被限制在 $r = a $和 $r = b$ 的两个不可穿透的同心球面之间运动,不存在其它势。求粒子的基态能量和归一化的波函数。

\banswer{
	
}

	
\end{enumerate}

