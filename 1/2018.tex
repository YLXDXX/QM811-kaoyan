\bta{2018}

\begin{enumerate}
	%\renewcommand{\labelenumi}{\arabic{enumi}.}
	% A(\Alph) a(\alph) I(\Roman) i(\roman) 1(\arabic)
	%设定全局标号series=example	%引用全局变量resume=example
	%[topsep=-0.3em,parsep=-0.3em,itemsep=-0.3em,partopsep=-0.3em]
	%可使用leftmargin调整列表环境左边的空白长度 [leftmargin=0em]
	\item
	质量为$ m $的粒子在一维无限深方势阱 $(0<\mathrm{x}<\mathrm{a})$ 中运动, 设初始波函数 $\psi(x, 0)=A\left[\phi_{1}(x)+\phi_{2}(x)\right]$, 其中 $\phi_{1}(x), \phi_{2}(x)$ 分别表示归一化的基态和第一激发态波函数。
	\begin{enumerate}
		%\renewcommand{\labelenumi}{\arabic{enumi}.}
		% A(\Alph) a(\alph) I(\Roman) i(\roman) 1(\arabic)
		%设定全局标号series=example	%引用全局变量resume=example
		%[topsep=-0.3em,parsep=-0.3em,itemsep=-0.3em,partopsep=-0.3em]
		%可使用leftmargin调整列表环境左边的空白长度 [leftmargin=0em]
		\item
		写出 $\phi_{1}(x), \phi_{2}(x)$ 和归一化因子$ A $。
		\item 
		求$ t=0 $时的 $\psi(x, t)$。
		\item 
		计算$ t=0 $时粒子坐标算符x和动量算符p的平均值。
		\item 
		计算$ t>0 $时粒子能量的可取值,对应概率和平均值。
		
		
		
	\end{enumerate}
	
	\banswer{
		
	}
	
\item 
考虑 $\mid lm>$ 态矢量空间中 $l=1$ 的子空间。在 $\left(\hat{l}, \hat{l}_{x}\right)$ 表象中角动量分量算符的矩阵表示为:
$$
\mathrm{L}_{\mathrm{x}}=\frac{\hbar}{\sqrt{2}}=\left[\begin{array}{lll}
	0 & 1 & 0 \\
	1 & 0 & 1 \\
	0 & 1 & 0
\end{array}\right], \mathrm{L}_{\mathrm{y}}=\frac{\hbar}{\sqrt{2}}=\left[\begin{array}{ccc}
	0 & -\mathrm{i} & 0 \\
	\mathrm{i} & 0 & -\mathrm{i} \\
	0 & \mathrm{i} & 0
\end{array}\right], \mathrm{L}_{\mathrm{z}}=\hbar=\left[\begin{array}{ccc}
	1 & 0 & 0 \\
	0 & 0 & 0 \\
	0 & 0 & -1
\end{array}\right]
$$
\begin{enumerate}
	%\renewcommand{\labelenumi}{\arabic{enumi}.}
	% A(\Alph) a(\alph) I(\Roman) i(\roman) 1(\arabic)
	%设定全局标号series=example	%引用全局变量resume=example
	%[topsep=-0.3em,parsep=-0.3em,itemsep=-0.3em,partopsep=-0.3em]
	%可使用leftmargin调整列表环境左边的空白长度 [leftmargin=0em]
	\item
	求 $\mathrm{L}_{\mathrm{x}}$ 矩阵的本征值和本征矢;
	\item 
	求联系 $\left(\hat{l}^{2}, \hat{l}_{z}\right)$表象和$\left(\hat{l}^{2}, \hat{l}_{x}\right)$ 表象的么正变换矩阵 $\mathrm{S}$;
	\item 
	利用幺正变换矩阵 $\mathrm{S}$, 求出 $\left(\hat{l}, \hat{l}_{x}\right)$ 表象中的矩阵表示 $\hat{l}_{x}, \hat{l}_{y}, \hat{l}_{z}$ 。
	
	
	
\end{enumerate}
\banswer{
	
}

\item 
一个磁矩为 $\vec{\mu}=\mu_{0} \vec{\sigma}$ 的自旋 $1 / 2$ 粒子在 $\mathrm{t}=0$ 时处于沿 $\mathrm{z}$ 正方向的均匀磁场 $\overrightarrow{\mathrm{B}}_{0}$
中, 其中$ \overrightarrow{\sigma} $为 Pauli 算符。当 $\mathrm{t}>0$ 时再加上一个旋转磁场 $\vec{B}_{1}(\mathrm{t})$, 其方向和 $z$ 轴垂	
直: $\overrightarrow{\mathrm{B}}_{1}(t)=\mathrm{B}_{1} \cos \left(2 \omega_{0} t \right) \overrightarrow{\mathrm{e}}_{x}-\mathrm{B}_{1} \sin \left(2 \omega_{0} t\right) \overrightarrow{\mathrm{e}}_{y}$, 其中 $\omega_{0}=\frac{\mu_{0} B_{0}}{\hbar}$ 。
\begin{enumerate}
	%\renewcommand{\labelenumi}{\arabic{enumi}.}
	% A(\Alph) a(\alph) I(\Roman) i(\roman) 1(\arabic)
	%设定全局标号series=example	%引用全局变量resume=example
	%[topsep=-0.3em,parsep=-0.3em,itemsep=-0.3em,partopsep=-0.3em]
	%可使用leftmargin调整列表环境左边的空白长度 [leftmargin=0em]
	\item
	求 $t>0$ 时的哈密顿算符 $\mathrm{H}=-\vec{\mu} \cdot\left[\overrightarrow{\mathrm{B}}_{0}+\overrightarrow{\mathrm{B}}_{1}({t})\right]$ 的本征值。 
	\item 
	若当 $t=0$ 时刻测得其自旋向下,求 $t$ 时刻自旋向上的概率。
	
	
	
	
\end{enumerate}
\banswer{
	
}


\newpage
\item 
一个质量为 $m$ 带电为$ -e  $的粒子在 $X-Y$ 平面内,做半径为 $r$ 的圆周运动。
\begin{enumerate}
	%\renewcommand{\labelenumi}{\arabic{enumi}.}
	% A(\Alph) a(\alph) I(\Roman) i(\roman) 1(\arabic)
	%设定全局标号series=example	%引用全局变量resume=example
	%[topsep=-0.3em,parsep=-0.3em,itemsep=-0.3em,partopsep=-0.3em]
	%可使用leftmargin调整列表环境左边的空白长度 [leftmargin=0em]
	\item
	求该粒子做圆周运动的能量本征值和本征函数。
	\item 
	 若在外场中添加一个微扰电场 $\overrightarrow{\mathscr{E}}=-\mathscr{E} \overrightarrow{\mathrm{e}}_{x}$,运用微扰理论, 求第一、第二激 发能量的二级近似。
	
	
	
\end{enumerate}

\banswer{
	
}


\item 
质量为 $m$ 的粒子在长度为 $a$ 的一维无限深对称坌势阱 $(|x|<a / 2)$ 中处于
基态。设 $t=0$ 时刻突然添加一高度为 $c$, 宽度为 $b(<a)$ 的方形微扰势 $H^{\prime}$, 在 $t=T$ 时刻突然撤去该微扰势。
\begin{enumerate}
	%\renewcommand{\labelenumi}{\arabic{enumi}.}
	% A(\Alph) a(\alph) I(\Roman) i(\roman) 1(\arabic)
	%设定全局标号series=example	%引用全局变量resume=example
	%[topsep=-0.3em,parsep=-0.3em,itemsep=-0.3em,partopsep=-0.3em]
	%可使用leftmargin调整列表环境左边的空白长度 [leftmargin=0em]
	\item
	写出一维无限深对称方势阱哈密顿算符 $\mathrm{H}_{0}$ 的本征值和归一化本征函数。
	\item 
	写出 $\mathrm{t}>\mathrm{T}$ 时刻态矢量 $|\varphi(\mathrm{t})\rangle$ 的表达式。
	\item 
	试用一级微扰近似计算在 $t>\mathrm{T}$ 后体系跃迁到第一、第二、第三激发态的概率,并算出当微扰势的宽带$ b $取何值时对应跃迁概率取最大值。
	
	
	
\end{enumerate}

\banswer{
	
}


\end{enumerate}

