
\bta{2005}

\begin{enumerate}
	%\renewcommand{\labelenumi}{\arabic{enumi}.}
	% A(\Alph) a(\alph) I(\Roman) i(\roman) 1(\arabic)
	%设定全局标号series=example	%引用全局变量resume=example
	%[topsep=-0.3em,parsep=-0.3em,itemsep=-0.3em,partopsep=-0.3em]
	%可使用leftmargin调整列表环境左边的空白长度 [leftmargin=0em]
	\item
($20'$)1800个电子经$1000V$电势差加速后从$x=-\infty$处射向势阶$V(x)=\begin{cases}V_0,&x<0\\0,&x>0\end{cases}$,其中$V_0=750V$。试问在$x=\infty$处能观察到多少个电子?如果势阶翻转一下,即电子射向势阶$V(x)=\begin{cases}0,&x<0\\V_0,&x>0\end{cases}$,则结果如何?


\banswer{
	
}


\item 
($20'$)质量为$m$,电荷为$q$的粒子在三维各向同性谐振子势$V(r)=\frac{1}{2}m\omega^2r^2$中运动,同时受到一个沿$x$方向的均匀常电场$\vec{E}=E_0\vec{i}$作用。求粒子的能量本征值和第一激发态的简并度。此时轨道角动量是否守恒?如回答是,则请写出此守恒力学量的表达式。


\banswer{
	
}


\item 
($40'$)一个质量为$m$的粒子在下面的无限深方势阱中运动。$V(x)=\begin{cases}\infty,&x<0,x>a\\ 0,&a>x>0\end{cases}$开始时$(t=0)$,系统处于状态,$\psi(x)=A\sin\frac{\pi x}{2a}\cos^3\frac{\pi x}{2a}$,其中$A$为常数。请求出$t$时刻系统:
\begin{enumerate}
	%\renewcommand{\labelenumi}{\arabic{enumi}.}
	% A(\Alph) a(\alph) I(\Roman) i(\roman) 1(\arabic)
	%设定全局标号series=example	%引用全局变量resume=example
	%[topsep=-0.3em,parsep=-0.3em,itemsep=-0.3em,partopsep=-0.3em]
	%可使用leftmargin调整列表环境左边的空白长度 [leftmargin=0em]
	\item
处于基态的几率;
\item 
能量平均值;
\item 
动量平均值;
\item 
动量的均方差根(不确定度)。
	
\end{enumerate}


\banswer{
	
}


\newpage


\item 
($30'$)两个具有相同质量$m$和频率$\omega$的谐振子,哈密顿量为
$$H^0=\frac{1}{2m}(p_1^2+p_2^2)+\frac{1}{2}m\omega^2((x_1-a)^2+(x_2+a)^2)$$
($\pm a$为两个谐振子的平衡位置),受到微扰作用$H'=\lambda m\omega^2(x_1-x_2)^2,|\lambda|\ll1$,试求该体系的能级。


\banswer{
	
}


\item 
($30'$)已知氢原子基态波函数为:$\psi_{100}=\frac{1}{(\pi a_0^3)^{\frac{1}{2}}}e^{-\frac{r}{a_0}}$,试对坐标$x$及动量$p_x$,求:
$\Delta x=\sqrt{\langle x^2\rangle-\langle x^2\rangle},\Delta p=\sqrt{\langle p_x^2\rangle-\langle p_x\rangle^2}$,由此验证不确定关系。

\banswer{
	
}


\item 
考虑自旋 $\vec{s}$ 与角动量 $\hat{\vec{L}}$ 的耦合, 体系的哈密顿量为
\[
\hat{H}=-\frac{\hbar^{2}}{2 \mu} \nabla^{2}+V(r)+\lambda \hat{\vec{L}} \cdot \hat{\vec{S}}
\]
$\lambda$ 是耦合常数, 试证该体系的总角动量 $\hat{\vec{J}}=\hat{\vec{L}}+\hat{\overrightarrow{\mathrm{S}}}$ 守恒。\\
公式提示:在球坐标系内,
\[ 
\nabla^{2}=\frac{1}{r^{2}}\left(\frac{\partial}{\partial r}\left(r^{2} \frac{\partial}{\partial r}\right)-\frac{\vec{L}^{2}}{\hbar^{2}}\right), \quad \nabla f(r)=\frac{\vec{r}}{r} \frac{\partial}{\partial r} f(r), \quad \int_{0}^{\infty} t^{n} e^{-t} d t=n !
 \]

\banswer{
	
}


\end{enumerate}


