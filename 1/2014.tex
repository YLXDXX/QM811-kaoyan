\bta{2014}

\begin{enumerate}
	%\renewcommand{\labelenumi}{\arabic{enumi}.}
	% A(\Alph) a(\alph) I(\Roman) i(\roman) 1(\arabic)
	%设定全局标号series=example	%引用全局变量resume=example
	%[topsep=-0.3em,parsep=-0.3em,itemsep=-0.3em,partopsep=-0.3em]
	%可使用leftmargin调整列表环境左边的空白长度 [leftmargin=0em]
	\item
	已知算符 $A=\left(\begin{array}{cc}1 & 0 \\ 0 & \mathrm{c}-1\end{array}\right), \quad \mathrm{B}=\left(\begin{array}{cc}0 & -i \\ i & 0\end{array}\right)$,其中c为实常数,当 $\mathrm{c}=\mathrm{c}_{0}$, 有共同本征态。
	\begin{enumerate}
		%\renewcommand{\labelenumi}{\arabic{enumi}.}
		% A(\Alph) a(\alph) I(\Roman) i(\roman) 1(\arabic)
		%设定全局标号series=example	%引用全局变量resume=example
		%[topsep=-0.3em,parsep=-0.3em,itemsep=-0.3em,partopsep=-0.3em]
		%可使用leftmargin调整列表环境左边的空白长度 [leftmargin=0em]
		\item
		求 $c_{0}$ 的值;
		\item 
		求当 $\mathrm{c}=\mathrm{c}_{0}$ 时算符 $A$ 和B共同本征态;
		\item 
		求当 $\mathrm{c} \neq \mathrm{c}_{0}$, 由 $A$ 表象到 $\mathrm{B}$ 表象的变换矩阵。
		
		
		
	\end{enumerate}


\banswer{
\begin{enumerate}
	%\renewcommand{\labelenumi}{\arabic{enumi}.}
	% A(\Alph) a(\alph) I(\Roman) i(\roman) 1(\arabic)
	%设定全局标号series=example	%引用全局变量resume=example
	%[topsep=-0.3em,parsep=-0.3em,itemsep=-0.3em,partopsep=-0.3em]
	%可使用leftmargin调整列表环境左边的空白长度 [leftmargin=0em]
	\item
	$ c_{0}=2 $
	\item 
	$ \ket{1}= \frac{1}{\sqrt{2}}\commamatrix[b]{1; i}$;$ \ket{-1} = \frac{1}{\sqrt{2}}\commamatrix[b]{1; -i}$
	\item 
	过渡矩阵为
	\[ 
	S=\frac{1}{\sqrt{2}}\commamatrix[b]{1,-i; 1,i}
	 \]
	\begin{note}
		由于基矢量的选择问题,过渡矩阵不唯一。
	\end{note}
\end{enumerate}


}

	
\item 
两个质量为 $\mu$ 的非全同粒子被紧闭在 $0 \leq x \leq L$ 的无限深势阱中。
\begin{enumerate}
	%\renewcommand{\labelenumi}{\arabic{enumi}.}
	% A(\Alph) a(\alph) I(\Roman) i(\roman) 1(\arabic)
	%设定全局标号series=example	%引用全局变量resume=example
	%[topsep=-0.3em,parsep=-0.3em,itemsep=-0.3em,partopsep=-0.3em]
	%可使用leftmargin调整列表环境左边的空白长度 [leftmargin=0em]
	\item
忽略两个粒子间的相互作用,求系统的三个最低能量和相应的归一化波函数;
\item 	
假设粒子间的相互作用 $\mathrm{V}=\lambda \delta\left(x_{1}-x_{2}\right)$ 的微弱相互作用, 求系统的三个最低能量($ \lambda $的一级近似)及相应的归一化波函数。
\end{enumerate}

\banswer{
\begin{enumerate}
	%\renewcommand{\labelenumi}{\arabic{enumi}.}
	% A(\Alph) a(\alph) I(\Roman) i(\roman) 1(\arabic)
	%设定全局标号series=example	%引用全局变量resume=example
	%[topsep=-0.3em,parsep=-0.3em,itemsep=-0.3em,partopsep=-0.3em]
	%可使用leftmargin调整列表环境左边的空白长度 [leftmargin=0em]
	\item
	两粒子的波函数和能量为
	\[ 
	\psi(x_{1},x_{2})=\frac{2}{L} \sin \frac{n_{1}\pi}{L} x_{1} \sin \frac{n_{2}\pi}{L} x_{2}  \quad E=\frac{\pi^{2}\hbar^{2}}{2mL^{2}}(n_{1}^{2}+n_{2}^{2})
	 \]
	基态波函数无简并,第一激发态二重简并,第二激发态无简并。
	\item 
	基态:$ E_{0}=\frac{\pi^{2}\hbar^{2}}{mL^{2}}+\frac{3}{2}\cdot \frac{\lambda}{L} $,波函数存疑。\\
	第一激发态:分裂成两条。\\
	第一条$ E_{11}= \frac{5}{2} \frac{\pi^{2}\hbar^{2}}{mL^{2}} $,$ \psi ^{\prime} _{11}=\frac{1}{\sqrt{2}}( \psi_{11} - \psi_{12})  $\\
	 	第一条$ E_{11}= \frac{5}{2} \frac{\pi^{2}\hbar^{2}}{mL^{2}} + 2\cdot \frac{ \lambda}{L} $,$ \psi ^{\prime} _{12}=\frac{1}{\sqrt{2}}( \psi_{11} + \psi_{12})  $
	
	
\end{enumerate}

	
}

\newpage
\item 
一个质量为 $\mu$、电荷为$ q $、自旋为$ 0 $的粒子被限制在 $xOy$ 平面内半径为 $a$ 的圆周上运动。
\begin{enumerate}
	%\renewcommand{\labelenumi}{\arabic{enumi}.}
	% A(\Alph) a(\alph) I(\Roman) i(\roman) 1(\arabic)
	%设定全局标号series=example	%引用全局变量resume=example
	%[topsep=-0.3em,parsep=-0.3em,itemsep=-0.3em,partopsep=-0.3em]
	%可使用leftmargin调整列表环境左边的空白长度 [leftmargin=0em]
	\item
	求该粒子的哈密顿量$ \hat{H} $及角动量算符的 $\mathrm{L}_{z}$ 的本征值及相应的归一化本征函数;

	\item 
	若在$ Z $轴加一磁场 $  \vv{B}  $, 求系统哈密顿量$ \hat{H} $的本征值及相应的归一化本征函数, 讨论加
	入磁场后能级简并度的变化。(提示取矢势$ \vv{\mathrm{A}}=\frac{1}{2} \vec{B} \times \vec{r} $)
	
	
\end{enumerate}

\banswer{
\begin{enumerate}
	%\renewcommand{\labelenumi}{\arabic{enumi}.}
	% A(\Alph) a(\alph) I(\Roman) i(\roman) 1(\arabic)
	%设定全局标号series=example	%引用全局变量resume=example
	%[topsep=-0.3em,parsep=-0.3em,itemsep=-0.3em,partopsep=-0.3em]
	%可使用leftmargin调整列表环境左边的空白长度 [leftmargin=0em]
	\item
	$ \hat{H}=\frac{L_{z}^{2}}{2 \mu a^{2}} $与$\mathrm{L}_{z}$有共同本征函数
	\[ 
	\psi_{m}(\phi)=\frac{1}{\sqrt{2 \pi}} e^{i m \varphi}  \quad  E_{m}=\frac{\hbar^{2} m^{2}}{2 \mu a^{2}}
	 \]
	$m=0, \pm 1, \pm 2, \cdots$,
	除了 $m=0$ 以外, 其他能级均二重简并
	
	\item 
	加上磁场后 
	\[ 
\begin{aligned}
		\hat{H}=\frac{1}{2 \mu}\left(\hat{\vec{p}}-\frac{q}{c} \vec{A}\right)^{2}&=\frac{1}{2 \mu}\left[\hat{\vec{p}}^{2}+\left(\frac{q}{c}\right)^{2} \vec{A}^{2}-\frac{q}{c}(\hat{\vec{p}} \cdot \vec{A}+\vec{A} \cdot \hat{\vec{p}})\right]\\
	&=\frac{1}{2 \mu}\left[\frac{L_{z}{ }^{2}}{a^{2}}+\left(\frac{q}{c}\right)^{2} \frac{a^{2} B^{2}}{4}-\frac{q}{c} B L_{z}\right]
\end{aligned}
	 \]
能量为
\[ 
E_{m}=\frac{1}{2 \mu}\left[\frac{(m \hbar)^{2}}{a^{2}}+\left(\frac{q}{c}\right)^{2} \frac{a^{2} B^{2}}{4}-\frac{q}{c} B m \hbar\right]
 \]	
简并解除
\end{enumerate}

	
}


\item 
一个质量为 $\mu$ 、电荷为 $q$ 的谐振子在外电场 $\varepsilon$ 的作用下 $ \hat{ \mathrm{H}}=\frac{p^{2}}{2 \mu}+\frac{1}{2} \mu \omega^{2} x^{2}-q \varepsilon x $。
\begin{enumerate}
	%\renewcommand{\labelenumi}{\arabic{enumi}.}
	% A(\Alph) a(\alph) I(\Roman) i(\roman) 1(\arabic)
	%设定全局标号series=example	%引用全局变量resume=example
	%[topsep=-0.3em,parsep=-0.3em,itemsep=-0.3em,partopsep=-0.3em]
	%可使用leftmargin调整列表环境左边的空白长度 [leftmargin=0em]
	\item
	若已知 $\hat{H_{0}}=\frac{P^{2}}{2 \mu}+\frac{1}{2} \mu \omega^{2} x^{2}$ 的本征能量为 $\mathrm{E}_{\mathrm{n}}$ 的相应本征函数为 $ \ket{n} $, 求 $\hat{\mathrm{H}}$ 的  本征
	能量 $\widetilde{\mathrm{E}}_{\mathrm{n}}$及相应的本征函数 $ \ket{\tilde{n}} $。
	\item 
	若 $\mathrm{t}=0$ 时刻处于系统 $\hat{H_{0}}$ 的初态 $ \ket{0} $, 求 $\mathrm{t}>0$ 时任处 $\hat{H_{0}}$ 的初态 $ \ket{0} $ 几率 $P_{0}$。
	
\end{enumerate}

\banswer{
\begin{enumerate}
	%\renewcommand{\labelenumi}{\arabic{enumi}.}
	% A(\Alph) a(\alph) I(\Roman) i(\roman) 1(\arabic)
	%设定全局标号series=example	%引用全局变量resume=example
	%[topsep=-0.3em,parsep=-0.3em,itemsep=-0.3em,partopsep=-0.3em]
	%可使用leftmargin调整列表环境左边的空白长度 [leftmargin=0em]
	\item
	$ \hat{H}=\frac{\hat{p}^{2}}{2 \mu}+\frac{1}{2} \mu \omega^{2}\left(\hat{x}-x_{0}\right)^{2}-\frac{q^{2} \varepsilon^{2}}{2 \mu \omega^{2}} $,其中 $x_{0}=\frac{q \varepsilon}{\mu \omega^{2}}$
	\[ 
	\widetilde{\mathrm{E}}_{\mathrm{n}}=\mathrm{E}_{\mathrm{n}}--\frac{q^{2} \varepsilon^{2}}{2 \mu \omega^{2}}  \qquad 
	\ket{\tilde{n}}=e^{-i\frac{x_{0}}{\hbar} \hat{p} } \ket{n}
	 \]
	\item 
	先将$ \ket{0} $用$ \ket{\tilde{n}} $表示出来,得到在$ t $时刻的波函数$ \ket{\tilde{\psi}(t)} $,再将$ \ket{\tilde{\psi}(t)} $用$ \ket{n} $表示出来,得到波函数$ \ket{\psi(t)} $,最后得到相应的概率
	\[ 
	P_{0}=e^{2 \cdot \frac{\omega}{\hbar} \cdot mx^{2}_{0} \cdot \sin \frac{\omega}{2} t}
	 \]
	
\end{enumerate}

	
}

\item 
三个电子做一维运动, 相互之间的作用可以忽略。单粒子的能量本征值为 $\varepsilon_{n}$, 相应的
归化一波函数为 $\psi_{n \sigma}=\psi_{n}(x) \chi_{\sigma} , \varepsilon_{0}<\varepsilon_{1}<\varepsilon_{2}<\ldots $,  $n=0,1,2 \ldots \ldots$; 其中 $\sigma$ 为单粒子自旋$ \hat{s} $的第三
分量 $\hat{s}_{z}$ 的本征态。
\begin{enumerate}
	%\renewcommand{\labelenumi}{\arabic{enumi}.}
	% A(\Alph) a(\alph) I(\Roman) i(\roman) 1(\arabic)
	%设定全局标号series=example	%引用全局变量resume=example
	%[topsep=-0.3em,parsep=-0.3em,itemsep=-0.3em,partopsep=-0.3em]
	%可使用leftmargin调整列表环境左边的空白长度 [leftmargin=0em]
	\item
	求该全同体系的基态能量及相应的归化一化本征函数。 
	\item 
	求系统总自旋的 $\hat{S}$的第三 分量 $\hat{S}_{z}=\hat{S}_{1 z}+\hat{S}_{2 z}+\hat{S}_{3 z}$ 在基态的平均值。
	
	
	
\end{enumerate}

\banswer{
\begin{enumerate}
	%\renewcommand{\labelenumi}{\arabic{enumi}.}
	% A(\Alph) a(\alph) I(\Roman) i(\roman) 1(\arabic)
	%设定全局标号series=example	%引用全局变量resume=example
	%[topsep=-0.3em,parsep=-0.3em,itemsep=-0.3em,partopsep=-0.3em]
	%可使用leftmargin调整列表环境左边的空白长度 [leftmargin=0em]
	\item
	基态能量$E_{0}=2 \varepsilon_{0}+\varepsilon_{1}$,二重简并
	\[ 
	\ket{\varphi_{1}}=  \frac{1}{\sqrt{6}}\left|\begin{array}{lll}
			\psi_{0+}(1) & \psi_{0+}(2) & \psi_{0+}(3) \\
			\psi_{0-}(1) & \psi_{0-}(2) & \psi_{0-}(3) \\
			\psi_{1+}(1) & \psi_{1+}(2) & \psi_{1+}(3)
		\end{array}\right|   \quad 
	\ket{\varphi_{2}}=\frac{1}{\sqrt{6}}\left|\begin{array}{lll}
		\psi_{0+}(1) & \psi_{0+}(2) & \psi_{0+}(3) \\
		\psi_{0-}(1) & \psi_{0-}(2) & \psi_{0-}(3) \\
		\psi_{1-}(1) & \psi_{1-}(2) & \psi_{1-}(3)
	\end{array}\right|
	 \]
	
\item 
$\left\langle\varphi_{1}\left|S_{z}\right| \varphi_{1}\right\rangle=\frac{\hbar}{2}$,$\left\langle\varphi_{2}\left|S_{z}\right| \varphi_{2}\right\rangle=-\frac{\hbar}{2}$	
	
\end{enumerate}


}


\end{enumerate}

