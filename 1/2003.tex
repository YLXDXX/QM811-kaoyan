\bta{2003}


\begin{enumerate}
	%\renewcommand{\labelenumi}{\arabic{enumi}.}
	% A(\Alph) a(\alph) I(\Roman) i(\roman) 1(\arabic)
	%设定全局标号series=example	%引用全局变量resume=example
	%[topsep=-0.3em,parsep=-0.3em,itemsep=-0.3em,partopsep=-0.3em]
	%可使用leftmargin调整列表环境左边的空白长度 [leftmargin=0em]
	\item
\begin{enumerate}
	%\renewcommand{\labelenumi}{\arabic{enumi}.}
	% A(\Alph) a(\alph) I(\Roman) i(\roman) 1(\arabic)
	%设定全局标号series=example	%引用全局变量resume=example
	%[topsep=-0.3em,parsep=-0.3em,itemsep=-0.3em,partopsep=-0.3em]
	%可使用leftmargin调整列表环境左边的空白长度 [leftmargin=0em]
	\item
	如果厄密算符 $\hat{A}$ 对任何矢量 $|u\rangle$, 有 $\langle u|\hat{A}| u\rangle \geq 0$, 则称 $\hat{A}$ 为正定算符, 求证:算符 $\hat{A}=|a\rangle\langle a|$ 是厄密正定算符。
	\banswer{
		
	}
	
	\item 
	如果 $\hat{A}$ 是任 一线性算符, 求证 $A^{+} A$ 是正定的厄密算符, 它的迹等于 $\hat{A}$ 在任意表象中的矩阵元的模平方之和。试推导, 当且仅当 $\hat{A}=0$ 时, $\operatorname{Tr}\left(A^{+} A\right)=0$ 才成立。
	
	\banswer{
		
	}
	
	
	\item 
	求证:如果 $[[\hat{A}, \hat{B}], \hat{A}]=[[\hat{A}, \hat{B}], \hat{B}]=0$, 则 $e^{\hat{A}} e^{\hat{B}}=e^{\hat{A}+\hat{B}} e^{\frac{1}{2}[\hat{A}, \hat{B}]}$。
	\banswer{
		
	}
	
	\item 
	求证:任一可观测的平均值对时间的导数由下式给出:
	\[
	i \hbar \frac{d\langle\hat{A}\rangle}{d t}=\langle[\hat{A}, \hat{H}]\rangle+i \hbar\left\langle\frac{\partial \hat{A}}{\partial t}\right\rangle
	\]
	\banswer{
		
	}



	
	
	
\end{enumerate}


	
	\item 
	把传导电子限制在金属内势的一种平均势, 对于下列一维模式(如图)
\[ 
	V(x)=\left\{\begin{array}{cc}
	-V_{0}, & x<0 \\
	0, & x>0
\end{array}\right. 
 \]
	试就:
	\begin{enumerate}
		%\renewcommand{\labelenumi}{\arabic{enumi}.}
		% A(\Alph) a(\alph) I(\Roman) i(\roman) 1(\arabic)
		%设定全局标号series=example	%引用全局变量resume=example
		%[topsep=-0.3em,parsep=-0.3em,itemsep=-0.3em,partopsep=-0.3em]
		%可使用leftmargin调整列表环境左边的空白长度 [leftmargin=0em]
		\item
 $E>0$,
	\item 
	$-V_{0}<E<0$ 
	\end{enumerate}
两种情况计算接近金属表面的传导电子的反射和透射几率。

\banswer{
	
}



\newpage
\item 
对于一维谐振子, 求消灭算符 $a$ 的本征态, 将其表示成各能量本征态 $|n\rangle$ 的线性叠加。

\banswer{
	
}

\item 
给定 $(\theta, \varphi)$ 方向单位矢量 $\vec{n}=\left(n_{x}, n_{y}, n_{z}\right)=(\sin \theta \cos \varphi, \sin \theta \sin \varphi, \cos \theta)$。求 $\sigma_{n}=\vec{\sigma} \cdot \vec{n}$ 的本征值和本征函数。(取 $\sigma_{z}$ 表象)


\banswer{
	
}

\item 
有一个定域电子(不计及其轨道运动)受到均匀磁场作用,磁场 $B$ 指向正 $x$ 方向, 磁作用势为:
\[
\hat{H}=\frac{e B}{\mu c} \cdot \hat{S}_{x}=\frac{e \hbar B}{2 \mu c} \hat{\sigma}_{x}
\]
设 $t=0$ 时电子的自旋 “向上”, 即 $S_{x}=\hbar / 2$, 求 $t>0$ 时 $\vec{S}$ 的平均值。

\banswer{
	
}


\end{enumerate}


