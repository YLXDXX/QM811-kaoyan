\bta{2015}


\begin{enumerate}
	%\renewcommand{\labelenumi}{\arabic{enumi}.}
	% A(\Alph) a(\alph) I(\Roman) i(\roman) 1(\arabic)
	%设定全局标号series=example	%引用全局变量resume=example
	%[topsep=-0.3em,parsep=-0.3em,itemsep=-0.3em,partopsep=-0.3em]
	%可使用leftmargin调整列表环境左边的空白长度 [leftmargin=0em]
	\item
	质量为 $\mu$ 的粒子限制在长度为 $L$ 的一维匣子 $(0<x<L)$ 中自由运动, 在 $x=0$ 与 $x=L$ 处其定态波函数满足 条件 $\Psi(0)=\Psi(L),  \Psi^{\prime}(0)=\Psi^{\prime}(L)$。
	\begin{enumerate}
		%\renewcommand{\labelenumi}{\arabic{enumi}.}
		% A(\Alph) a(\alph) I(\Roman) i(\roman) 1(\arabic)
		%设定全局标号series=example	%引用全局变量resume=example
		%[topsep=-0.3em,parsep=-0.3em,itemsep=-0.3em,partopsep=-0.3em]
		%可使用leftmargin调整列表环境左边的空白长度 [leftmargin=0em]
		\item
		求体系能级;
		\item
		将第一激发态归一化波函数表示为动量本征态的线性组合,并求动量平均值$ \bar{p} $为$ 0 $时组合系数满足的条件。
		
		
		
		
	\end{enumerate}
	
\banswer{
\begin{enumerate}
	%\renewcommand{\labelenumi}{\arabic{enumi}.}
	% A(\Alph) a(\alph) I(\Roman) i(\roman) 1(\arabic)
	%设定全局标号series=example	%引用全局变量resume=example
	%[topsep=-0.3em,parsep=-0.3em,itemsep=-0.3em,partopsep=-0.3em]
	%可使用leftmargin调整列表环境左边的空白长度 [leftmargin=0em]
	\item
	可将方程的解表示为
	\[ 
	\psi(x)=A \sin kx + B \cos kx
	 \]
	这里最终只能得到$ kL=2n\pi $,系数$ A $和$ B $定不出来,说明这个体系的能量本征态存在简并情况。另外,关于$ n $的取值范围为$ n=0,\pm 1, \pm 2 ,\pm 3,,\cdots $。能量为
	\[ 
	E_{n}=\frac{\pi^{2} \hbar^{2}}{2 m L^{2}}(2 n)^{2}
	 \]
\item 
第一激发态$ n=\pm 1 $,$ k=\pm \frac{2\pi}{L} $,利用欧拉关系式,波函数可为
\[ 
\psi(x)=c_{1} e^{ikx} + c_{2}e^{-ikx}
 \]
得到系数关系为$\left|c_{1}\right|=\left|c_{2}\right|=\frac{1}{\sqrt{2}}$。	
\end{enumerate}


}

\item 
粒子在球对称谐振子势阱 $V(r)=\frac{1}{2} \mu \omega^{2}\left(x^{2}+y^{2}+z^{2}\right)$ 中运动, 受到微扰作用
\[ 
H^{\prime}=\lambda\left(x y z+x^{2} y+x y^{2}\right)
 \]
其中 $\lambda$ 为常数。求准确到二级微扰修正的基态能量。\\
提示:粒子数表象下, $ \left\langle n_{x}^{\prime}|\hat{x}| n_{x}\right\rangle=\sqrt{\frac{\hbar}{2 \mu \omega}}\left(\sqrt{n_{x}+1} \delta_{n_{x}^{\prime}, n_{x}+1}+\sqrt{n_{x}} \delta_{n_{x}^{\prime}, n_{x}-1}\right) )$。
	
\banswer{
\begin{enumerate}
	%\renewcommand{\labelenumi}{\arabic{enumi}.}
	% A(\Alph) a(\alph) I(\Roman) i(\roman) 1(\arabic)
	%设定全局标号series=example	%引用全局变量resume=example
	%[topsep=-0.3em,parsep=-0.3em,itemsep=-0.3em,partopsep=-0.3em]
	%可使用leftmargin调整列表环境左边的空白长度 [leftmargin=0em]
	\item
	设$ \ket{\psi}=\ket{n_{x}n_{y}n_{z}} $,$ E_{n}=\left(n_{x}+n_{y}+n_{z}+\frac{3}{2}\right) \hbar \omega $。\\
	 ①显然$ \braket{000|H ^{\prime} |000}=0 $\\
	②矩阵元$ \braket{n_{x}n_{y}n_{z}|H ^{\prime} |000}$不为零的情况如下
	\[ 
	\begin{tblr}{c}
		\hline
		\braket{111|xyz|000}& \braket{010|x^{2}y |000}&\braket{210|x^{2}y|000} & \braket{001|xy^{2} |000} & \braket{120|xy^{2}|000}\\
		\hline
	\end{tblr}
	 \]
基态能量的二级修正为$ E= \frac{3}{2}\hbar\omega -\frac{11(\lambda \hbar)^{2}}{24 \mu^{3} \omega^{4}} $	
\end{enumerate}


}


\item 
两个自旋为 $\frac{1}{2}$ 的粒子组成的体系, $\hat{\boldsymbol{S}}_{\mathbf{1}}$ 和 $\hat{\boldsymbol{S}}_{\mathbf{2}}$ 分别表示两个粒子的自旋算符, $\hat{\boldsymbol{S}}$ 为两粒子的总自旋算符, $\hat{\boldsymbol{n}}$ 表 示两粒子相对运动方向的单位矢量。设系统的相互作用哈密顿量为 $H=3\left(\hat{\boldsymbol{S}}_{\mathbf{1}} \cdot \hat{\boldsymbol{n}}\right)\left(\hat{\boldsymbol{S}}_{\mathbf{2}} \cdot \hat{\boldsymbol{n}}\right)-\hat{\boldsymbol{S}}_{1} \cdot \hat{\boldsymbol{S}}_{2}$。
\begin{enumerate}
	%\renewcommand{\labelenumi}{\arabic{enumi}.}
	% A(\Alph) a(\alph) I(\Roman) i(\roman) 1(\arabic)
	%设定全局标号series=example	%引用全局变量resume=example
	%[topsep=-0.3em,parsep=-0.3em,itemsep=-0.3em,partopsep=-0.3em]
	%可使用leftmargin调整列表环境左边的空白长度 [leftmargin=0em]
	\item
	证明:
	\begin{enumerate}
		%\renewcommand{\labelenumi}{\arabic{enumi}.}
		% A(\Alph) a(\alph) I(\Roman) i(\roman) 1(\arabic)
		%设定全局标号series=example	%引用全局变量resume=example
		%[topsep=-0.3em,parsep=-0.3em,itemsep=-0.3em,partopsep=-0.3em]
		%可使用leftmargin调整列表环境左边的空白长度 [leftmargin=0em]
		\item
		$\hat{\boldsymbol{S}}_{1} \cdot \hat{\boldsymbol{S}}_{2}=\frac{1}{2} \hat{\boldsymbol{S}}^{2}-\frac{3}{4} \hbar^{2}$
		\item 
		$\left(\hat{\boldsymbol{S}}_{\mathbf{1}} \cdot \hat{\boldsymbol{n}}\right)\left(\hat{\boldsymbol{S}}_{\mathbf{2}} \cdot \hat{\boldsymbol{n}}\right)=\frac{1}{2}(\hat{\boldsymbol{S}} \cdot \hat{\boldsymbol{n}})^{2}-\frac{1}{4} \hbar^{2}$
		
		
		
	\end{enumerate}
	
	
\item 
证明 $\left[H, \hat{\boldsymbol{S}}^{\mathbf{2}}\right]=0$	
	
	
\end{enumerate}


\banswer{
\begin{enumerate}
	%\renewcommand{\labelenumi}{\arabic{enumi}.}
	% A(\Alph) a(\alph) I(\Roman) i(\roman) 1(\arabic)
	%设定全局标号series=example	%引用全局变量resume=example
	%[topsep=-0.3em,parsep=-0.3em,itemsep=-0.3em,partopsep=-0.3em]
	%可使用leftmargin调整列表环境左边的空白长度 [leftmargin=0em]
	\item
	证明略。其中会遇到$ (\hat{S_{1}} \cdot \hat{n}  )^{2} $的表达式。处理这个有两种方法:①利用$ S_{1}=\frac{\hbar}{2} \sigma_{1} $;②利用$ \hat{S_{1}} \cdot \hat{n}  $的本征值。得到
	\[ (\hat{S_{1}} \cdot \hat{n}  )^{2} = \frac{1}{4} \hbar^{2} \]
	
	\item 
	证明略
	
\end{enumerate}

	
}



\newpage
\item 
质量为 $\mu$ 的粒子在一维势 $V(x)=\left\{\begin{array}{ll}\infty & x<0 \\ B x & x>0\end{array}\right.$ 中运动, 其中 $B>0$ 是常数。
\begin{enumerate}
	%\renewcommand{\labelenumi}{\arabic{enumi}.}
	% A(\Alph) a(\alph) I(\Roman) i(\roman) 1(\arabic)
	%设定全局标号series=example	%引用全局变量resume=example
	%[topsep=-0.3em,parsep=-0.3em,itemsep=-0.3em,partopsep=-0.3em]
	%可使用leftmargin调整列表环境左边的空白长度 [leftmargin=0em]
	\item
	试从下列波函数中选择一个合理的束缚态试探波函数, 并说明理由。
\threechoices
{$e^{-\frac{x}{a}}$}
{$x e^{-\frac{x}{a}}$}
{$1-e^{-\frac{x}{a}}$}
其中 $a>0$ 为变分参数。

\item 
取所选的试探波函数,用变分法估算体系基态能量。

	
	
	
\end{enumerate}


\banswer{
\begin{enumerate}
	%\renewcommand{\labelenumi}{\arabic{enumi}.}
	% A(\Alph) a(\alph) I(\Roman) i(\roman) 1(\arabic)
	%设定全局标号series=example	%引用全局变量resume=example
	%[topsep=-0.3em,parsep=-0.3em,itemsep=-0.3em,partopsep=-0.3em]
	%可使用leftmargin调整列表环境左边的空白长度 [leftmargin=0em]
	\item
	选$x e^{-\frac{x}{a}}$,理由略。
	\item 
	先归一化:
	\[ 
	\psi=\left\{\begin{aligned}
		&0 & x<0 \\
		&\frac{2}{\sqrt{a^{3}}} x e^{-x / a} & x \geq 0
	\end{aligned}\right.
	 \]
	得到
	\[ 
	E=\frac{\hbar^{2}}{2 \mu a^{2}}+B \frac{3 a}{2}
	 \]
	 当$ a=\left(\frac{2 \hbar^{2}}{3 B \mu}\right)^{1 / 3} $时$\frac{\partial E}{\partial a}=0$,此时$E=\frac{9 }{4}\cdot B \cdot \left(\frac{2 \hbar^{2}}{3 B \mu}\right)^{1 / 3}$
\end{enumerate}


}



\item 
一个二能级体系, 哈密顿算符的矩阵表达式为 $H_{0}=\left(\begin{array}{cc}E_{1}^{(0)} & 0 \\ 0 & E_{2}^{(0)}\end{array}\right),\left(E_{1}^{(0)}<E_{2}^{(0)}\right) $。 设 $t=0$ 时刻体系处
于 $H_{0}$ 的基态上, 后受微扰 $H^{\prime}$ 的作用, $H^{\prime}=\left(\begin{array}{cc}0 & \gamma \\ \gamma & 0\end{array}\right), \gamma$ 为常数。
\begin{enumerate}
	%\renewcommand{\labelenumi}{\arabic{enumi}.}
	% A(\Alph) a(\alph) I(\Roman) i(\roman) 1(\arabic)
	%设定全局标号series=example	%引用全局变量resume=example
	%[topsep=-0.3em,parsep=-0.3em,itemsep=-0.3em,partopsep=-0.3em]
	%可使用leftmargin调整列表环境左边的空白长度 [leftmargin=0em]
	\item
	求出在 $t>0$ 时刻, 体系处于 $H_{0}$ 激发态的几率 $P_{E_{2}^{(0)}}(t)$ 的精确表达式。
	\item 
	利用一阶含时微扰论求 $P_{E_{2}^{(0)}}(t)$, 并与精确表达式比较, 讨论所得结果的适用条件。
	
	
	
\end{enumerate}


\banswer{
\begin{enumerate}
	%\renewcommand{\labelenumi}{\arabic{enumi}.}
	% A(\Alph) a(\alph) I(\Roman) i(\roman) 1(\arabic)
	%设定全局标号series=example	%引用全局变量resume=example
	%[topsep=-0.3em,parsep=-0.3em,itemsep=-0.3em,partopsep=-0.3em]
	%可使用leftmargin调整列表环境左边的空白长度 [leftmargin=0em]
	\item
	可用含时薛定谔方程解,或者利用$ H=H_{0}+H ^{\prime}  $的本征态来解。
	\[ 
	H=\commamatrix[b]{E_{1}^{(0)},\gamma; \gamma,E_{2}^{(0)}}
	 \]
本征值为
\[ 
\hspace*{-1cm}
\varepsilon_{1}=\frac{E_{1}^{(0)}+E_{2}^{(0)} - \sqrt{\left(E_{1}^{(0)}-E_{2}^{(0)}\right)^{2}+4\gamma^{2}}}{2}  \quad \varepsilon_{2}=\frac{E_{1}^{(0)}+E_{2}^{(0)} + \sqrt{\left(E_{1}^{(0)}-E_{2}^{(0)}\right)^{2}+4\gamma^{2}}}{2}
 \]
$ \varepsilon_{1} $对应的本征态为$ \varphi_{1} $,$ \varepsilon_{2} $对应的本征态为$ \varphi_{2} $
\[ 
\varphi_{1}=\left(\begin{aligned}
	\frac{\gamma}{\sqrt{\left(\varepsilon_{1}-E_{1}^{(0)}\right)^{2}+\gamma^{2}}} \\
	\frac{\varepsilon_{1}-E_{1}^{(0)}}{\sqrt{\left(\varepsilon_{1}-E_{1}^{(0)}\right)^{2}+\gamma^{2}}}
\end{aligned}\right)
 \qquad 
 \varphi_{2}=\left(\begin{aligned}
 	\frac{\gamma}{\sqrt{\left(\varepsilon_{2}-E_{1}^{(0)}\right)^{2}+\gamma^{2}}} \\
 	\frac{\varepsilon_{2}-E_{1}^{(0)}}{\sqrt{\left(\varepsilon_{2}-E_{1}^{(0)}\right)^{2}+\gamma^{2}}}
 \end{aligned}\right)
 \]
初态$ \ket{\psi_1}=  \commamatrix[b]{1; 0} $随时间演化为
\[ 
\ket{\psi(t)}=a e^{-i \frac{\varepsilon_{1}}{2} t } \varphi_{1}  +be^{-i \frac{\varepsilon_{2}}{2} t } \varphi_{2}
 \]
其中$ a=\braket{ \varphi_{1}|\psi_1 } $,$ b=\braket{ \varphi_{2}|\psi_1 } $。处于$ H_{0} $的激发态$ \ket{\psi_{2}}=\commamatrix[b]{0; 1} $的概率为
\[ 
\begin{aligned}
P_{E_{2}^{(0)}}(t)=&\braket{\psi_{2}|\psi(t)} \\
=&\left|
\frac{\gamma}{\sqrt{\left(\varepsilon_{1}-E_{1}^{(0)}\right)^{2}+\gamma^{2}}} \cdot
\frac{\varepsilon_{1}-E_{1}^{(0)}}{\sqrt{\left(\varepsilon_{1}-E_{1}^{(0)}\right)^{2}+\gamma^{2}}} 
\cdot 
e^{-i \frac{\varepsilon_{1}}{2} t }
\right. 
+
\\
&\left.
\frac{\gamma}{\sqrt{\left(\varepsilon_{2}-E_{1}^{(0)}\right)^{2}+\gamma^{2}}} \cdot
\frac{\varepsilon_{2}-E_{1}^{(0)}}{\sqrt{\left(\varepsilon_{2}-E_{1}^{(0)}\right)^{2}+\gamma^{2}}}
\cdot
e^{-i \frac{\varepsilon_{2}}{2} t }
\right|^{2}
\end{aligned}
\]

\item 
利用一阶含时微扰论计算 $P_{E_{2}^{(0)}}(t)$的结果为
\[ 
P_{E_{2}^{(0)}}(t) = \frac{4 \gamma ^{2}}{(E_{2}^{(0)} -E_{1}^{(0)} )^{2}} \sin^{2} \left( \frac{E_{2}^{(0)} -E_{1}^{(0)}}{2\hbar} \right)t
 \]

\end{enumerate}

	
}


\end{enumerate}


